\article{}

\section{The Executive Officers}

\subsection{}
The President will preside at meetings of the Assembly, oversee and coordinate all MSA activities, and be the chief spokesperson for MSA unless otherwise specified in the Code or Constitution.  The President may appoint a chair to preside over any portion of an Assembly meeting.  The President must appoint a Treasurer, Student General Counsel, and Chief of Staff prior to the end of the Winter Semester.

\subsection{}
The Executive Vice President will chair the Assembly in the absence of the President, coordinate and supervise the general administrative staff of MSA, work with the Administrative Coordinator to assign MSA office space, preside at meetings of the Steering Committee, maintain a record of all activities of the Assembly, and prepare an MSA Mid-Year Report and MSA End-Year Report.

\subsection{}
The Treasurer will chair the Assembly in the absence of the President and Executive Vice President, prepare one budget for the Fall term and one budget for Winter/Spring/Summer terms in consultation with the other executive officers, committee, commission, and select commission chairs, the Administrative coordinator, and the MSA financial advisor. The Treasurer shall promptly disburse allocated funds with the assistance of the Administrative Coordinator, promptly respond to requests for budgetary information by MSA members The Treasurer shall make available any information concerning the budget or finance information of MSA available upon request to representatives and committee and commission chairs. The Front Office shall make available the Annual Budget to available upon request to any student. The Administrative Coordinator shall make available upon request any previously prepared financial information to any student. 

\subsection{}
The Student General Counsel will chair the Assembly in the absence of the President, Executive Vice President and Treasurer, will serve as MSA counsel before student judiciaries with  full authority in representing MSA before such judiciaries, will serve as Parliamentarian of the Assembly, will assist MSA members who desire technical assistance in drafting or amending MSA legislation, will in consultation with legal experts report legal advice to the Assembly as needed, will review MSA legislation and motions for compliance with the Code and Constitution, and will be responsible for assessing absences to representatives and chairs as outlined in the Code and Constitution.

\subsection{}
The Chief-of-Staff will, work with the Administrative Coordinator to assign MSA officer space, prepare the agenda for Assembly meetings, maintain a record of the actions of the Assembly's committees and commissions, and assist the Executive Vice-President in the preparation of an MSA Mid-Year Report and MSA End-Year Report.


\section{Steering Committee}

\subsection{Meetings.}  The Steering Committee shall meet weekly before Assembly meetings during the fall and winter semesters.  Additional meetings of the Committee may be scheduled by the Executive Vice President with 24 hours advance notice to the members of the Committee.

\subsection{Authority.}
\subsubsection{}
All business for the Assembly must be presented to the Steering Committee.  All such business will be presented to the Assembly unless ruled out of order by the Student General Counsel.
\subsubsection{}
The Steering Committee may, without Assembly approval, authorize the expenditure of up to \$250 from any MSA account by a motion, a second and two-thirds vote.
\subsubsection{}
All actions of the Steering Committee may be overturned at the subsequent meeting of the Assembly by a motion, second, and majority vote.


\section{Central Student Judiciary}

\subsection{Supremacy Clause.}
This section is superseded by the MSA Constitution but supersedes all other sections of the Compiled Code and all other MSA legislation with respect to judiciaries and judicial proceedings.  Amendments to this chapter must be approved by CSJ.

\subsection{Central Student Judiciary.}
CSJ will serve as the principle \todo{typo} student judiciary and as the judicial branch of the all-campus student government provided for in Article X of the MSA Constitution.  It has primary responsibility for enforcing the MSA Constitution and for adjudicating disputes arising out of the MSA Constitution, Compiled Code and other legislation enacted pursuant to the MSA Constitution.

\subsection{Jurisdiction.}
CSJ has jurisdiction over actions where there is no other judicial body with jurisdiction or where it is not clear which judiciary has jurisdiction.  CSJ has original jurisdiction in all disputes concerning which body shall hear a particular action.

\subsection{CSJ Structure and Membership.}
See also MSA Constitution: X, A and B.
\subsubsection{}
The officers of CSJ (Chief Justice, Associate Chief Justice, Administrative Justice) will be elected from among the members of CSJ following the appointment of new members each term.  The Associate Chief Justice will serve as Acting Chief Justice if the Chief Justice is unable to perform a duty required of the office.  If both the Chief Justice and the Associate Chief Justice are unable to perform the duties required of the Chief Justice, CSJ will elect an Acting Chief Justice from its membership to serve until either the Chief Justice or Associate Chief Justice is able to serve or until the next election of officers.
\subsubsection{}
Except where specifically provided for elsewhere in this section, courts of CSJ may have partially or completely overlapping memberships.
\subsubsection{}
If an action raises a conflict of interest for a member of CSJ, that member cannot hear the action, either in CSJ's original jurisdiction or on appeal.  No member of CSJ, including the Chief Justice, who heard an action at trial, may hear an appeal on the action.
\subsubsection{}
The Chief Justice can fill any vacancies in any CSJ court from the membership of CSJ as necessary.

\subsection{CSJ Procedures.}
This section, as well as the CSJ Manual of Judicial Procedure and CSJ Manual of Administrative Procedure, will form the Manual of Procedure mandated in the MSA Constitution.  The CSJ Manual of Judicial Procedure will govern all judicial proceedings before any court of CSJ.

\subsection{CSJ Courts.}

\subsubsection{}
General Hearing Courts have original jurisdiction in each action within the jurisdiction of CSJ except for those specifically within the jurisdiction of an Election Court.  A new General Hearing Court is created each time a case arises and serves until the case is disposed.  The General Hearing Court consists of three CSJ members, one of whom will be the President Justice of the court.  The Chief Justice of CSJ appoints members of the General Hearing Court and designates the Presiding Justice.  The Chief Justice can serve on the court and can designate him or herself Presiding Justice.

\subsubsection{}
Election Courts have the powers given to the ``Election Board'' in the MSA Constitution (note that this is not the same as the ``Election Board'' constituted by MSA in the section on ``Election Code'').  The Election Court has jurisdiction over any action arising out of MSA general or special elections.  The Election Court has jurisdiction over all actions arising under the ``Election Code'', the ``Code on Petitions and Ballot Questions'', and the section on ``Seat Reapportionment''.  A new Election Court is created for each election.  The Election Court consists of three members of CSJ, one of whom will be the Presiding Justice of the court.  The Chief Justice of CSJ has the same powers over the Election Court as s/he does over the General Hearing Courts.

\subsubsection{}
Appellate Courts have appellate jurisdiction in each action within the jurisdiction of CSJ.  A new Appellate Court is appointed by the Chief Justice of CSJ each time a case arises, and serves until disposition of the case.  The Appellate Court consists of all members of CSJ who are not disqualified or unable to serve for other reasons and in no case less than three justices.  One of the justices will be designated as the Presiding Justice of the court.  If the Chief Justice of CSJ serves on the court, he or she can serve as the Presiding Justice; otherwise the court will select a Presiding Justice from its membership.


\section{Student Governments.}

\subsection{Definition.}
A student government must be democratically constituted in accordance with Article I(C) of the Constitution and must hold college- or school-wide elections on at least an annual basis.  Only one government may represent the entirety of any one school or college.

\subsection{Appointments to MSA.}  

\subsubsection{}
School and college governments shall have the right to make appointments to fill vacant MSA seats apportioned to that school or college.

\subsubsection{}
No school or college government appointment to MSA is valid without written confirmation to the MSA executive officers from the presiding officer of the school or college government.

\subsubsection{}
An appointment to MSA shall endure only  for the remainder of the term of the vacated seat.

\subsection{Rights and Responsibilities.}

a. 	School and college governments shall have the right to register as student organizations.

b.	All school and college governments must register a copy of their constitution and by-laws, and any subsequent amendments, with MSA.

c.	All school and college governments must appoint a contact person to serve as a liaison between MSA and the school or college government, and must register this contact information with MSA.

\section{Student Organizations.}  

\subsection{Registration.}
A student group seeking registration with MSA must comply with all of the MSA rules and regulations required for student organizations.  Failure to comply with any regulation may result in a termination of the organization's status as a registered student organization.  A student group is registered automatically upon the receipt by the MSA Administrative Coordinator of a qualified application for registration.


\subsection{Requirements for Registration.}

\subsubsection{}
A student organization must have at least five currently-enrolled University of Michigan students as members.

\subsubsection{}
More than half of the total membership of the group must be students currently enrolled at the University of Michigan.

\subsubsection{}
At least two-thirds of the total group membership must be comprised of University of Michigan students, alumni, faculty or staff.

\subsubsection{}
No member of a student organization can receive personal financial benefit from membership in the organization.

\subsubsection{}
No organization can adopt a name which may be construed by the University community as misleading concerning the nature or affiliation of the organization.

\subsubsection{}
``The University of Michigan'' may not be used in the beginning of any student organization name.

\subsubsection{}
A new registration form is required for each school year.

\subsubsection{}
Groups must update MSA with new contacts and authorized signer information as changes occur.

\subsubsection{}
An application for registration must include a written description of the organization.

\subsection{Termination of Registration.}
The registration for all student organizations shall terminate at the end of September of every year.  Registration may also be terminated at any time if the group fails to meet the requirements for registration.  MSA shall have the responsibility of notifying an active student organization of impending termination of its registered status.  

\subsection{Ex-Officio Representation}

\subsubsection{}
A group wishing to attain an ex-officio seat shall submit a list of first name, last name, and email address of at least 400 members, as well as a signed statement acknowledging that they do not belong to a larger organization and are not a college or school student government on the MSA Website's online ex-officio submission tool.
\subsubsection{}
If a question is raised about the validity of the 400-member roster, the Rules and Elections Committee will conduct an investigation on the number of students in the student organization in question.
\subsubsection{}
Ex-Officio seats shall expire at the end of every winter semester. Groups wishing to re-apply to retain their seats in the fall shall retain their seat until a determination is made regarding their eligibility for the seat in the fall.
\subsubsection{}
There shall be no limit to the number of groups allowed to have ex-officio seats. All groups meeting the criteria shall be granted a seat.
\subsubsection{}
Student organization ex-officio members shall have all the rights of a regular assembly member, except they may not make motions, second a motion, or vote.


\section{External Organizations of Which MSA is a Member}

\subsection{The Association of Big Ten Students (``ABTS'').}
MSA shall send a delegation to ABTS conferences.  Membership in the delegation shall be subject to nomination by the External Relations Committee and a confirmation by two-thirds of the Steering Committee.   

\subsection{The Student Association of Michigan (``SAM'').}
MSA shall send a delegation to SAM meetings.  Membership in the delegation shall be subject to nomination by the External Relations Committee and a confirmation by two-thirds of the Steering Committee.


\section{Responsibilities of Appointees}

\subsection{Responsibilities of Appointees.}  Students appointed by the Assembly to any advisory or policy-making committee of the University must ensure that
\subsubsection{}
the Campus Governance Committee has the current contact and student status information of the appointee;
\subsubsection{}
the Campus Governance Committee receives information about the function and activities of the committee to which she was appointed;
\subsubsection{}
the appointee regularly attends meetings of the committee to which she was appointed.

\subsection{Political Responsibility of Appointees.}
Students appointed by the Assembly to any advisory or policy-making committee of the University may not take any position or cast any vote contrary to the Constitution, Code, or any article of MSA legislation.

\subsection{Recall of Appointees.}
The Campus Governance Committee may recall any student appointed to any University committee for failure to discharge her responsibilities.  A recall shall be effective upon a motion, second, and two-thirds vote of the Steering Committee.


\section{MSA Legislation}

\subsection{}
All business proposed to the Assembly shall be ruled out of order by the Student General Counsel if not in compliance with the Constitution or Compiled Code.

\subsection{Additional Requirements for MSA Legislation.}

\subsubsection{Sponsor.}  MSA legislation shall be out of order if it is not sponsored by at least one member of MSA.  Students at large, committees, commissions, select committees, student governments, and student organizations may all be sponsors, but are not members of MSA.

\subsubsection{Action.}  MSA legislation shall be deemed out of order by the Student General Counsel if it is purely symbolic or does not specify an action to be taken by MSA.

\subsubsection{Truthful Statements.}  MSA legislation shall be out of order if it contains untrue or unverifiable statements of fact.

\subsubsection{Financial Specificity.}  MSA legislation authorizing expenditure or transfer of MSA money for events may do so only from MSA Committee Discretionary or MSA Sponsored Activities accounts.  Resolutions shall be out of order if it does not specify the MSA account(s) to be debited, the recipient(s) of the funds, the purpose(s) of the expenditure(s), and the exact amount(s) to be spent. Legislation authorizing expenditure or transfer of MSA money from MSA Sponsored Activities must be in accordance with IV I.

\subsubsection{Informative Requirement.}  MSA legislation shall be out of order if it is so vague that it does not provide enough information to the Assembly to make an informed decision.

\subsubsection{}
All MSA legislation must have line numbers (at least every five lines).  If lines are not numbered, the legislation shall be ruled out of order by the Student General Counsel.

\subsection{Obligation to Execute MSA Legislation.}  No committee, commission, select committee, or member of MSA may take any action in its or her capacity as an MSA affiliate which contradicts any article of MSA legislation.

\subsection{Duration of MSA Legislation.} 

\subsubsection{}
MSA legislation excluding funding requests shall remain effective until repealed.
\subsubsection{}
Funding requests shall remain effective until the last day of classes of the Winter Semester of the academic year they were passed.


\section{Parliamentary Rights.}
Parliamentary rights on the Assembly shall be granted as follows:

\subsection{}
Assembly members shall be granted full parliamentary rights on the Assembly.

\subsection{}
All other officers elected or confirmed by the Assembly shall be considered members of MSA and shall be granted all parliamentary rights on the Assembly except for the right to vote. Members of CSJ, the Election Director and Election Board, appointees to University committees, and regular MSA committee, commission, or select committee members shall not be considered officers of MSA

\subsection{}
Individuals not accounted for above shall not be considered members of MSA and shall not be granted parliamentary rights on the Assembly, though they shall be permitted to speak during the time reserved for community concerns or when yielded to by an MSA member in debate.
