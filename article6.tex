\article{}

\section{Conflicts of Interest.}

\subsection{Conflicts of Interest with the University.}  
\subsubsection{}
No member of the Assembly may accept a paid position gained by virtue of her membership in MSA. 
\subsubsection{}
No member of the Assembly may work directly for the president, any Regent, any dean, any vice president, or any associate vice president of the University. 

\subsection{Misuse of Assembly Resources.}
\subsubsection{}
No member or employee of the Assembly may ask an employee of the Assembly to do work that is substantially unrelated to Assembly work.
\subsubsection{}
No member or employee of the Assembly may use the financial or clerical services of the Assembly for her personal use or for the use of a student group of which that person is a member.  


\subsection{Conflicts of Interest with Student Organizations.}
\subsubsection{}
A member of the Assembly shall have a conflict of interest with a student organization, be it recognized or unrecognized by the Assembly, if she, or an immediate family member of hers, receives money from the organization or will receive money from the organization as a direct consequence of her membership in the Assembly.  No member of the Assembly shall have a conflict of interest with a student organization if she is an unpaid member of the organization or has been reimbursed by the organization for her own expenses.
\subsubsection{}
No member of the Assembly possessing a conflict of interest with a student organization may participate in debate or vote on any matter regarding the organization with which there exists a conflict of interest.

\subsection{Obligation of Disclosure.}
Members of the Assembly must publicly disclose any existing or potential conflicts of interest with the University and student organizations.  If the conflict of interest arises during membership in the Assembly, the member remains obligated to disclose the conflict.

\subsection{Investigation, Censure, and Removal.}
Any member of the Assembly who fails to adhere to the rules and regulations regarding the existence and mitigation of conflicts of interest shall be subject to an investigation by the Assembly, the outcome of which may be censure or removal from the Assembly.


\section{Member Obligations.}

\subsection{}
Assembly members are responsible for compliance with the obligations established in Article XI of the Constitution.  The Vice-Chair of the Rules \& Elections Committee shall be responsible for recording the attendance of MSA members.

\subsection{Rules for Removal.}
Pursuant to Article XI(B) of the Constitution, an Assembly	 member shall be removed from MSA upon her/his accumulation of twelve absences.
\subsubsection{}
Upon the accumulation of twelve absences by any member, the Rules \& Elections Committee shall be responsible for notifying the delinquent member of her/his removal, and shall inform the delinquent member of the procedures for the excuse of absences.  The notification shall not be valid unless in writing and setting forth the date and type of each absence.
\subsubsection{}
A delinquent member may have her/his absence(s) excused within two weeks of the notification of removal.  During this time, the delinquent member shall retain all rights of membership in MSA.  If, after the expiration of this two-week period, the member is still delinquent, she/he shall be effectively removed.  No Assembly confirmation shall be necessary for such a removal to be effective.

\subsection{Rules for the Excuse of Absences.}
\subsubsection{}
a.	Any absence of any member may be excused if the excuse for the absence is one of those listed in Article XI(B) of the Constitution.  To be effective, the absence must be excused by
\subsubsubsection{}
The MSA President; or
\subsubsubsection{}
The Steering Committee, by a two-thirds vote; or
\subsubsubsection{}
The Assembly, by a majority vote upon a motion made during the Announcements portion of an Assembly meeting.

\subsubsection{}
Any absence of any member may be excused if the excuse for the absence is not listed in Article XI(B) of the Constitution only by a unanimous vote of the Assembly upon a motion made during the Announcements portion of an Assembly meeting.

\subsubsection{}
No absence of a member shall be excused after the member has been effectively removed from MSA.


\subsection{Student Group Outreach.}  

\subsubsection{}
The Vice-Chair of the Campus Governance Committee shall be responsible for compiling a list of student organizations with more than one hundred members, and for assigning MSA representatives to act as liaisons between MSA and these organizations.

\subsubsection{}
MSA representatives assigned to act as liaisons between MSA and a student organization must inform the organization of MSA activity at least once per month, and must inform MSA of the organization’s activity in the form of a general announcement at least once per month.

\subsubsection{}
Any MSA representative assigned to act as a liaison between MSA and a student organization who fails to meet the requirements of a liaison shall have one absence recorded for every month in which she fails to properly act as a liaison.


\section{Meeting Procedures.}  

\subsection{Assembly Meetings.}

\subsubsection{Regular Meetings.}  The Assembly shall meet at 7:30pm every Tuesday during the fall and winter semesters.  The Assembly shall not meet during exam and vacation periods.

\subsubsection{Special Meetings.}  The President may call special meetings of the Assembly upon providing at least 24 hours notice to the Assembly.  A special meeting may be called by a petition signed by one-third of the voting members of the Assembly delivered to the President at least 24 hours before the meeting.  No officer may be elected during a special meeting.

\subsection{Meeting Agenda.}

\subsubsection{}
The agenda will be in the following form:
\subsubsubsection{}
Call to Order
\subsubsubsection{}
Opening Roll Call
\subsubsubsection{}
Approval of Agenda
\subsubsubsection{}
Approval of Previous Minutes
\subsubsubsection{}
Guest Speakers
\subsubsubsection{}
Community Concerns
\subsubsubsection{}
Announcements
\subsubsubsection{}
Executive Officers’ Reports
\subsubsubsection{}
Committee \& Commission Reports
\subsubsubsection{}
Representative Reports
\subsubsubsection{}
Student Organization Funding Recommendations
\subsubsubsection{}
Campus Governance Committee Recommendations
\subsubsubsection{}
Motions to Veto Any Actions of the Steering Committee
\subsubsubsubsection{}
Attendance
\subsubsubsection{}
Election and Recall of Members
\subsubsubsection{}
Amendments to the Constitution or Compiled Code
\subsubsubsection{}
Old Business
\subsubsubsection{}
New Business
\subsubsubsection{}
Matters Arising
\subsubsubsection{}
Closing Roll Call
\subsubsubsection{}
Adjourn

\subsubsection{}
Committee reports will be given in the following order: Steering Committee and then internal committees, commissions, and select committees (in the order determined by the chiar).

\subsubsection{}
Representative Reports shall exist for the purpose of providing a forum for MSA Representatives to report on their MSA  projects and activities.  Each representative shall be required to give a report at least every three weeks.  The reporting schedule will be determined by the Chief of Staff.  The Chief of Staff must give representatives at least 72 hours advance notice.  Reports must be submitted in writing or be presented orally at an Assembly meeting.

\subsubsection{}
Community concerns is limited to five minutes per speaker and to a total of one hour.  Any person may ask to address the Assembly during this time period, but at the discretion of the chair, preference shall be given to currently enrolled students, alumni of the University, and current faculty and staff of the University.  All time limits may be extended by a majority vote of the Assembly, but community concerns time may not be reduced.  Community concerns time is exhausted when all persons who desired to address the Assembly have done so even if an hour has not passed.

\subsubsection{}
Business may be placed on the agenda by being presented to the Steering Committee, or by a two-thirds majority vote of the Assembly at any Assembly meeting.  Such a motion shall not be in order if made after the approval of the agenda by the Assembly.

\subsubsection{}
All business for the Assembly shall be read twice.  At the first reading, the sponsors of the business shall offer a description of the business.  At the second reading, the business shall be debated and voted upon.

\subsubsection{}
Old Business shall consist of all business postponed from previous meetings and all business being read a second time by the Assembly.  New Business shall consist of all business being read a first time by the Assembly. 

\subsubsection{}
New business may be moved to old business upon a motion, second, and two-thirds majority vote to suspend the Compiled Code for the purpose of the motion, followed by a motion, second, and two-thirds majority vote to move the item from old to new business.

\subsubsection{}
Amendments to the Constitution or Compiled Code, the MSA budget and amendments to the budget, and proposals to place questions on the ballot in an election may not be moved from old to new business, and must be read twice by the Assembly. Amendments to the MSA Constitution and proposals to place questions on the ballot in an election must return to first-reads should they be amended during second reads.

\subsubsection{}
Matters Arising, a time for anyone with parliamentary rights to speak about issue that have come up, may not last for more than 30 minutes total unless time is extended, and it can only be extended once and no person shall speak for more than two minutes at once.



\subsection{Procedures for Debate.}

\subsubsection{}
Robert’s Rules of Order shall govern the parliamentary procedure of the Assembly meetings.  The Compiled Code shall supercede Robert’s Rules of Order.

\subsubsection{Time Limits.}
\subsubsubsection{}
Debate on each item of business on the agenda shall be no longer than  thirty minutes, and shall include time used to debate amendments to business.
\subsubsubsection{}
Reports from committees, commissions, and select committees shall be no longer than five minutes for each report.
\subsubsubsection{}
Executive officer reports shall be no longer than ten minutes for each report.
\subsubsubsection{}
Guest speakers to the Assembly shall have no longer than thirty minutes to speak.
\subsubsubsection{}
During a debate on an item of business, each speaker shall have two minutes.
\subsubsubsection{}
Any time limit may be extended by a motion, second, and majority vote of the Assembly.  Any time limit may be reduced by a motion, second, and two-thirds majority vote of the Assembly.

\subsubsection{Yielding Time During Debate}
\subsubsubsection{}
Any Assembly member may yield her time to any constituent.  The time yielded to a constituent may not exceed the time which was allocated to the Assembly member.  Constituents may not yield time to other constituents.
\subsubsubsection{}
Any Assembly member may yield time to any other Assembly member for the purpose of asking a question germane to the debate.  The Assembly member may chose whom, when, and for how long they wish to yield to another member.  No member of the Assembly may yield more time than they were allocated.  Any member may reclaim the time they have yielded at any time.  Questions shall not count to the two speaking maximum per item.  If the member who has been yielded time for a question fails to ask a question germane to the item in a timely manner, the Student General Counsel shall issue a warning, and then revert the yielded time back to the original member.

\subsubsection{}
Assembly members may not proxy votes to other Assembly members.

\subsubsection{Motions to Close Debate.}
\subsubsubsection{}
A motion to close debate on a main motion shall not be in order until at least two members have spoken in favor of the motion and two members have spoken against the motion.  
\subsubsubsection{}
A motion to close debate on an amendment to a main motion shall not be in order until at least one member has spoken in favor of the motion, and one member has spoken against the motion.
\subsubsubsection{}
No more members may speak for or against any motion than have spoken for the opposing view.


\subsubsection{}
A call for quorum shall be in order once during the time limit for debate on any motion.  Members not present during a quorum call shall not be able to vote until the next motion is debated.

\subsubsection{}
The sponsors of a motion may accept amendments as “friendly” before or during debate on a motion, and such amendments shall be adopted without debate or vote by the Assembly.  When more than one member has sponsored a motion, every sponsor of the motion must accept an amendment for it to be considered “friendly”.

\subsubsection{}
If there are no objections to a motion to call the question, the Assembly shall proceed immediately to a vote.  If any member objects to a motion to call the question, the Assembly shall vote only upon a motion, second, and two-thirds majority vote of the Assembly.

\subsubsection{}
Assembly members may object to motions while simultaneously offering an amendment to the motion.  If the amendment to the motion is defeated, the objection shall be considered automatically withdrawn unless renewed by the objecting member.

\subsubsubsection{}
The Chair shall vote only to break a tie vote by the Assembly, but may abstain from voting in that instance.  The Chair shall not vote, even in the event of a tie, when the vote in question is by secret ballot.

\subsubsection{}
In the absence of unanimous consent, individual votes shall be recorded at all times for MSA legislation. The votes of individual members shall be recorded in the minutes of the meeting at which the vote occurred. This shall only apply to votes on the actual legislation itself and shall not apply to votes taken on any other motion made while said legislation is under consideration. 

\subsection{Minutes.}  A copy of the agenda, minutes, written reports, and final versions of both adopted and rejected legislation shall be kept in the MSA office.  All records shall be open to public examination in perpetuity.  No Assembly member may remove the minutes from the MSA office.


\section{Hiring Procedures.}  

\subsection{Full-Time Staff.}  The Administrative Coordinator and any MSA employee paid for more than twenty-five hours of work per week by MSA shall be considered full-time staff.  Hiring and termination of full-time staff shall be governed by the University’s Standard Practice Guide for employees.

\subsection{Administrative Staff.}  All paid employees and organizations of MSA who are not full-time staff shall be administrative staff.

\subsubsection{Hiring.}  Administrative staff positions may be created by the Executive Vice President, who shall interview and hire administrative staff members with the assistance of the Administrative Coordinator.

\subsubsection{Termination.}  Upon the request by the Administrative Coordinator or Executive Vice President, and a motion, second, and two-thirds majority vote, the Steering Committee may terminate the employment of any administrative staff member.


\subsection{Spring-Summer Employees.}  The procedures for the creation, hiring, and termination of staff positions shall persist during the spring and summer semesters, unless the Assembly adopts different procedures no later than the last regular Assembly meeting of the winter semester.  No hiring or termination conducted during the spring or summer semester shall be permanent until confirmed by the Assembly during the fall or winter semester.
