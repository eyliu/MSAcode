\article{Finance}

\section{Semesterly Budget}

\subsection{Revenue.}
MSA will collect revenue from student fees, its balance carry-forward from the previous semester, and interest income from the University investment pool.  


\subsection{Accounts.}

\subsubsection{General Account.}
The MSA General Account shall include all MSA revenue.  Money from this account will be transferred to other accounts upon the adoption of the annual budget.

\subsubsection{General Reserve.}
The MSA General Reserve account shall be used for emergency funding if necessary.  The amount budgeted to the General Reserve from the General Account by the annual budget shall be at least 5\% of projected incoming revenue from student fees.  No money may be allocated from the General Reserve without the approval of two-thirds of the Assembly.

\subsubsection{Operations.}
The MSA Operations Account shall be used for funding the management of the MSA offices, but shall not be used to pay full-time or part-time staff.  The amount budgeted to the Operations Account from the General Account by the annual budget shall no more than 4\% of projected incoming revenue from student fees.

\subsubsection{Payroll.}
The MSA Payroll Account shall be used to pay MSA's full-time and part-time staff.  The amount budgeted to the Payroll Account from the General Account by the annual budget may not exceed 20\% of projected incoming revenue from student fees.

\subsubsection{Childcare.}
\$1.00 per student per semester shall be collected by Student Financial Operations and transferred to a University Financial Aid account for the maintenance of childcare for students.  This money shall not pass into or through any of MSA's accounts.

\subsubsection{Committee and Commission Accounts.}
Each MSA committee, commission, and select committee shall have its own account.  Unless otherwise instructed herein, the Treasurer shall determine the amount to be budgeted to each of these accounts in the annual budget.
\subsubsubsection{Budget Priorities Committee.}
The amount budgeted to the Budget Priorities Committee Account from the General Account by the semesterly budget shall be at least 30\% of projected incoming revenue from student fees or \$75,000, whichever is greater.
\subsubsubsection{Community Service Commission.}
The amounted budgeted to the Community Service Commission Account from the General Account by the semesterly budget shall be at least 12\% of projected incoming revenue from student fees or \$30,000, whichever is greater.
\subsubsubsection{}
Funds from Budget Priorities Committees and Community Service Commission are the only funds that may be transferred to registered student organization SOAS accounts.

\subsubsection{Committee Discretionary.}
The MSA Committee Discretionary shall be used by the Assembly for expenditures on specific projects and tasks of committee, commissions, and select committees. The amount budgeted to the Committee Discretionary account from the General Account by the annual budget shall consist of all funds not budgeted to any other MSA account. Committee Discretionary funds may only be disbursed to valid MSA committee, commission, or select committee SOAS accounts.

\subsection{Enacting the Budget.}
The Treasurer shall prepare the annual budget in consultation with the executive officers and with the assistance of the administrative coordinator and MSA financial advisor. Committee, commission and select committee chairs must submit a budget proposal to the Treasurer for review no later than two (2) weeks into the fall term or no later than one (1) week prior to the start of the winter term. The Treasurer shall propose the two (2) term budgets to the Assembly no later than the third meeting of the Fall and Winter terms. It shall be approved upon a motion, a second and a majority vote.

\subsection{Amendments to the Budget.}
The annual budget may be amended by the Assembly by a motion, second, and majority vote.

\subsection{}
The Treasurer must hold an open meeting on each term's proposed budget at least seven (7) days before the vote on the budget is set to be voted upon.


\section{Disbursements.}

\subsection{Committee, Commission, and Select Committees.}
\subsubsection{}
The chair of a committee, commission, or select committee may spend money from that committee, commission or select committee account only with the consent and signature of an executive officer. Committee, commission and select committee chairs may only spend money allocated to them on the specific projects that the money was allocated for.
\subsubsubsection{}
\$150 with the consent and signature of an executive officer;
\subsubsubsection{}
\$250 with the consent and signature of an executive officer and the a two-thirds vote of the Steering Committee;
\subsubsubsection{}
any amount with the consent and signature of an executive officer and the approval of a majority of the Assembly.
\subsubsection{}
\$100 from each committee, commission, and select committee account may be used by the chair for making copies, and need not require approval of an executive officer.  If this \$100 is exhausted, additional funds from that committee, commission, or select committee account may be used for copies with the consent of an executive officer.
\subsubsection{Reimbursement.}
Upon presentation of the appropriate receipts, the Administrative Coordinator will reimburse the committee, commission, or select committee chair for expenditures.  The amount reimbursed shall not exceed the amount allocated to the committee or commission, and must conform to the conditions under which the expenditure was approved.
\subsubsection{}
Alterations to an individual committee, commission or select committee budget after the MSA budget has been passed through the budget enacting process must be approved by the Executive Officers.

\subsection{Operations Disbursements.}
Any executive officer, by her consent and signature, may authorize the allocation of up to \$250 from the Operations account for supplies without the prior authorization of the Steering Committee or the Assembly.

\subsection{Payroll Disbursements.}
\subsubsection{}
Any executive officer, by her consent and signature, may authorize the disbursement of salary from the Payroll Account to part-time MSA staff without the prior authorization of the Steering Committee or the Assembly.  
\subsubsection{}
The Director or Assistant Director of the Office of Student Activities and Leadership, by her consent and signature, may authorize the disbursement of salary from the Payroll Account to full-time MSA staff without prior approval of the Steering Committee or the Assembly.

\subsection{Authorized Account Signatures.}
\subsubsection{}
Expenditures from any MSA account shall require two authorized signatures.  All MSA executive officers and the MSA Administrative Coordinator shall be authorized to approve expenditures from every MSA account.  Committee, commission, and select committee chairs shall be authorized to approve expenditures only from that committee, commission, or select committee account.  Persons acting as chair shall not be authorized to approve expenditures from any account.  
\subsubsection{}
The Steering Committee or the Assembly, upon a motion, second, and a vote, may authorize any person to approve expenditures from any MSA account.

\subsection{Disbursements to External Organizations.}
Per Article VI, Section H of the All-Campus Constitution, no disbursement from any MSA account to an external organization shall be approved without a majority vote of the Assembly. 
