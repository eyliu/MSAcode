\article{}

\section{Annual Budget}

\subsection{Revenue.}
MSA will collect revenue from student fees, its balance carry-forward from the previous year, and interest income from the University investment pool.  

\subsection{Accounts.} 

\subsubsection{General Account.}  The MSA General Account shall include all MSA revenue.  Money from this account will be transferred to other accounts upon the adoption of the annual budget.

\subsubsection{General Reserve.}  The MSA General Reserve account shall be used for emergency funding if necessary.  The amount budgeted to the General Reserve from the General Account by the annual budget shall be at least 5\% of projected incoming revenue.  No money may be allocated from the General Reserve without the approval of two-thirds of the Assembly.


\subsubsection{Operations.}  The MSA Operations Account shall be used for funding the management of the MSA offices, but shall not be used to pay full-time or part-time staff.  The amount budgeted to the Operations Account from the General Account by the annual budget shall be at least 4\% of projected incoming revenue.


\subsubsection{Payroll.}
The MSA Payroll Account shall be used to pay MSA's full-time and part-time staff.  The amount budgeted to the Payroll Account from the General Account by the annual budget may not exceed 15\% of projected incoming revenue.


\subsubsection{Childcare.}
\$1.00 per student per semester shall be collected by Student Financial Operations and transferred to a University Financial Aid account for the maintenance of childcare for students.  This money shall not pass into or through any of MSA's accounts.


\subsubsection{Committee and Commission Accounts.}
Each MSA committee, commission, and select committee shall have its own account.  Unless otherwise instructed herein, the Treasurer shall determine the amount to be budgeted to each of these accounts in the annual budget.

\subsubsubsection{Budget Priorities Committee.}
The amount budgeted to the Budget Priorities Committee Account from the General Account by the annual budget shall be at least 40\% of projected incoming revenue or \$175,000, whichever is greater.

\subsubsubsection{Community Service Commission.}
The amounted budgeted to the Community Service Commission Account from the General Account by the annual budget shall be at least 20\% of projected incoming revenue or \$87,500, whichever is greater.

\subsubsubsection{}
Funds from Budget Priorities Committees and Community Service Commission are the only funds that may be transferred to registered student organization SOAS accounts.


\subsubsection{Committee Discretionary.}
The MSA Committee Discretionary shall be used by the Assembly for expenditures on specific projects and tasks of committee, commissions, and select committees. The amount budgeted to the Committee Discretionary account from the General Account by the annual budget shall consist of all funds not budgeted to any other MSA account. Committee Discretionary funds may only be disbursed to valid MSA committee, commission, or select committee SOAS accounts.


\subsection{Enacting the budget.}
The Treasurer shall prepare the annual budget in consultation with the executive officers and with the assistance of the administrative coordinator and MSA financial advisor. Committee, commission and select committee chairs must submit a budget proposal to the Treasurer for review no later than two (2) weeks into the fall term or no later than one (1) week prior to the start of the winter term. The Treasurer shall propose the two (2) term budgets to the Assembly no later than the third meeting of the Fall and Winter terms. It shall be approved upon a motion, a second and a majority vote.


\subsection{Amendments to the Budget.}
The annual budget may be amended by the Assembly by a motion, second, and majority vote.

\subsection{}
The Treasurer must hold an open meeting on each term's proposed budget at least seven (7) days before the vote on the budget is set to be voted upon.


\section{Disbursements.}

\subsection{Committee, Commission, and Select Committees.}  

\subsubsection{}
The chair of a committee, commission, or select committee may spend money from that committee, commission or select committee account only with the consent and signature of an executive officer. Committee, commission and select committee chairs may only spend money allocated to them on the specific projects that the money was allocated for.
\subsubsubsection{}
\$150 with the consent and signature of an executive officer;
\subsubsubsection{}
\$250 with the consent and signature of an executive officer and the a two-thirds vote of the Steering Committee;
\subsubsubsection{}
any amount with the consent and signature of an executive officer and the approval of a majority of the Assembly.
\subsubsection{}
\$100 from each committee, commission, and select committee account may be used by the chair for making copies, and need not require approval of an executive officer.  If this \$100 is exhausted, additional funds from that committee, commission, or select committee account may be used for copies with the consent of an executive officer.

\subsubsection{Reimbursement.}
Upon presentation of the appropriate receipts, the Administrative Coordinator will reimburse the committee, commission, or select committee chair for expenditures.  The amount reimbursed shall not exceed the amount allocated to the committee or commission, and must conform to the conditions under which the expenditure was approved.

\subsubsection{}
Alterations to an individual committee, commission or select committee budget after the MSA budget has been passed through the budget enacting process must be approved by the Executive Officers.


\subsection{Operations Disbursements.}
Any executive officer, by her consent and signature, may authorize the allocation of up to \$250 from the Operations account for supplies without the prior authorization of the Steering Committee or the Assembly.


\subsection{Payroll Disbursements.}  

a.	Any executive officer, by her consent and signature, may authorize the disbursement of salary from the Payroll Account to part-time MSA staff without the prior authorization of the Steering Committee or the Assembly.  

b.	The Director or Assistant Director of the Office of Student Activities and Leadership, by her consent and signature, may authorize the disbursement of salary from the Payroll Account to full-time MSA staff without prior approval of the Steering Committee or the Assembly.

\subsection{Authorized Account Signatures.}  

a.	Expenditures from any MSA account shall require two authorized signatures.  All MSA executive officers and the MSA Administrative Coordinator shall be authorized to approve expenditures from every MSA account.  Committee, commission, and select committee chairs shall be authorized to approve expenditures only from that committee, commission, or select committee account.  Persons acting as chair shall not be authorized to approve expenditures from any account.  

b. 	The Steering Committee or the Assembly, upon a motion, second, and a vote, may authorize any person to approve expenditures from any MSA account.   

\subsection{Disbursements to External Organizations.}  Per Article VI, Section H of the All-Campus Constitution, no disbursement from any MSA account to an external organization shall be approved without a majority vote of the Assembly. 


\section{Office Space Allocation Committee (OSAC).}

\subsection{Purpose.}
The purpose of the Office Space Allocation Committee is to provide University of Michigan student organizations with criteria and applications for office space and locker usage.  OSAC shall reviews applications for space and allocate office space and lockers on the fourth floor of the Michigan Union. 

\subsection{Composition}

\subsubsection{}
OSAC will be composed of 8 student members.  These 8 members constitute the voting members of OSAC.  Quorum shall be a majority of voting committee members.  A simple majority shall be required for all committee decisions.

\subsubsection{}
3 OSAC members will represent the Michigan Union Board of Representatives (MUBR).  One of the three representatives must be the Chairperson of MUBR or her designee.

\subsubsection{}
3 OSAC members will represent the Michigan Student Assembly.  One of the three representatives must be the Vice President of MSA or her designee. 

\subsubsection{}
2 OSAC members will be at-large members.  The selection of these members is the duty of the Campus Governance Committee.

\subsubsection{}
In addition to the 8 voting members, the Administrative Coordinator of MSA and a Michigan Union representative will attend the meetings of OSAC as non-voting members.

\subsubsection{}
The MUBR Chairperson, the MSA Vice President, and the MSA Administrative Coordinator will jointly determine the weekly meeting time and place for OSAC.

\subsubsection{}
If an OSAC member is absent at more than two OSAC meetings, she will be removed from the committee and will automatically be replaced by appointment from the Campus Governance Committee.

\subsubsection{}
Two transition meetings between the old and new OSAC committees will be held.  The first meeting will take place within two weeks of the applications being made available.  The second meeting will occur during the first meeting of the new OSAC in which applications are reviewed.


\subsection{Internal Positions}

\subsubsection{}
The MSA Administrative Coordinator will serve as the chair of OSAC.  During all OSAC meetings, the chair will maintain order within the committee, keep the committee focused, and vote in the event of a tie.

\subsubsection{}
OSAC will appoint an Internal Secretary.  The Internal Secretary will record the minutes from every meeting and keep proper documentation of all activities.  The Administrative Coordinator shall maintain copies of all documentation.

\subsubsection{}
OSAC will also appoint an External Secretary.  The External Secretary will serve as a correspondent to all parties outside the committee.

\subsubsection{}
The Internal and External Secretaries will be elected by the committee through a simple majority of open voting.

\subsubsection{}
All OSAC members must complete a summary of each application they are assigned to review.  These summaries will be maintained by the MSA Administrative Coordinator.


\subsection{Process.}

\subsubsection{}
OSAC application materials shall be made available at the beginning of the winter semester.  

\subsubsection{}
Applications will be due one month after they are made available.

\subsubsection{}
OSAC may contact a student organization for more information or clarification of their application.

\subsubsection{}
No late applications will be accepted.  Student organizations which submit a late application will be notified immediately that their applications were not accepted.


\subsection{Appeals}

\subsubsection{}
Grounds for appeal will be limited to:
\subsubsubsection{}
deviations from the office space allocation procedure as set forth in this article.  
\subsubsubsection{}
penalties applied by MSA, MUBR or the Union Administration regarding office space that are arguably inappropriate for the violation.
\subsubsubsection{}
non-allocation of office space to a student organization who which correctly followed all of the application steps.

\subsubsection{}
The Appeals Board will be composed of 1 MUBR member (not included in the allocation process), 2 MSA members (not included in the allocation process), one Union Administration member (not included in the allocation process), and one student-at-large selected by the Campus Governance Committee.

\subsubsection{}
The composition of the Appeals Board will be determined within the first two weeks that appeals are made available.

\subsubsection{}
An appeal must be submitted in writing, with the president, chairperson, or equivalent's signature, to the MSA office no later than 5 business days after the original penalty was assessed.

\subsubsection{}
The Appeals Board will meet within 2 days of the appeals due date and determine whether the appeal has reason to be heard.

\subsubsection{}
If the Appeals Board finds a reason for appeals to be heard, appeals will take place over the following Saturday and Sunday.  Appeal sign-ups will be posted in MSA.

\subsubsection{}
The organization requesting the appeal can bring no more than 5 members to the appeal.

\subsubsection{}
Only oral presentations with a typed supplement will be considered at the Appeals hearing.

\subsubsection{}
The Appeals Board will decide on the appeal no later than 5 days after the conclusion of the meeting.  The Appeals Board can advise OSAC to reconsider the application, and can ask OSAC to meet with the members of the appealing organization for an information review.

\subsubsection{}
Deviations from the timeline by an appealing student organization will render the appeal null and void.


\section{Budget Priorities Committee (BPC)}

\subsection{}
BPC shall consider funding requests for all student organizations and their events under the guidelines established below.

\subsection{Membership Structure}
\subsubsection{}
BPC shall consist of a Chair, Vice-Chair, Reviews, and Appeals Board.
\subsubsection{Voting Membership}
\subsubsubsection{}
BPC shall be composed of a Reviews Board and an Appeals Board.
\subsubsubsubsection{}
Each Board must have at least four but no more than nine voting members.
\subsubsubsubsubsection{}
The Chair is non-voting member.
\subsubsubsubsubsection{}
The Vice-chair is a voting member.
\subsubsubsubsection{}
Each Board must maintain a majority of voting MSA Representatives.

\subsubsubsection{}
The Reviews Board and Appeals Board shall be recommended by the BPC Chair and shall be confirmed upon a motion, second and a two-thirds majority vote of the Steering Committee (SC) of MSA.
\subsubsubsection{Removal of Voting Members}
\subsubsubsubsection{}
Any voting member from either Board may be removed upon a motion, second, and two-thirds majority vote of the SC.
\subsubsubsubsection{}
The Chair of the SC shall be responsible for contacting the Board member removed with a written explanation of the reason for removal and shall place it in the SC minutes. 

\subsection{Voting Rights}
\subsubsection{}
No voting member from either Board may vote on a request for funds from any student organization that they hold an appointed, compensated, or elected leadership position in.

\subsubsection{}
Violations of paragraph (3.a) shall be grounds for immediate removal from either Board.

\subsubsection{}
Violations by members of MSA shall constitute malfeasance in office and be grounds for impeachment or removal from all offices and positions held in MSA.

\subsubsection{}
Prior to a vote related to the finances of an organization, members of either Board are required to declare any financial or personal interest they have with that organization.

\subsubsection{Chair Voting}
\subsubsubsection{}
The Chair shall vote to break a tie. 
\subsubsubsection{}
The Chair may not vote in any other circumstances.


\subsection{BPC Procedure}
\subsubsection{}
BPC shall determine and recommend funding allocations to the MSA on a viewpoint neutral basis. 

\subsubsection{}
BPC may not consider the membership, composition, or political views of any organization when deliberating funding recommendations. 

\subsubsection{}
Funding applications to BPC shall be made available to student organizations within two weeks of the start of each semester and shall remain available until the application deadline for the final cycle of that semester.

\subsubsection{}
BPC shall consider no more than one application per organization per cycle.

\subsubsection{}
Upon the request of an officer of a student organization, the BPC Chair, or designee, shall provide a written justification for that organization's recommended allocation.

\subsubsection{}
Upon the request of any member of MSA, the BPC Chair, or designee, shall provide a written justification for the recommended allocation of any organization. 

\subsubsection{}
Any money allocated to a student organization by the Assembly upon recommendation from BPC which is unspent by the organization shall be considered canceled by the organization and shall revert to MSA.

\subsubsection{}
The BPC Chair, with the assistance of the Administrative Coordinator, will oversee the disbursement and reimbursement process of student organizations from BPC earmarked funds.


\subsection{Student Organization Requirements}
\subsubsection{}
All student groups applying for funding must be registered with MSA and have a valid SOAS account.

\subsubsection{}
Student organizations must present accurate information to BPC through written applications and any oral statements.

\subsubsection{BPC conditions}
\subsubsubsection{}
BPC may attach any conditions to their allocations regarding the use of funds.

\subsubsubsection{}
Organizations receiving funding must stipulate in a grant agreement that they will adhere to these conditions.

\subsubsubsection{}
Failure to adhere to the conditions attached to the agreement by BPC shall result in a cancellation of the agreement, and all allocated funds shall revert to MSA. 

\subsubsubsection{}
BPC shall not fund, unless deemed necessary by a two-thirds majority vote of the committee:
\subsubsubsubsection{}
Capital goods
\subsubsubsubsection{}
T-shirts
\subsubsubsubsection{}
Newspaper advertisements
\subsubsubsubsection{}
Hotel or airfare costs for students traveling from campus
\subsubsubsubsection{}
Gas
\subsubsubsubsection{}
Club sports fees assessed by the Athletic Department

\subsubsubsection{}
Organizations receiving funding from BPC must agree to either include the phrase “Sponsored by the Michigan Student Assembly” or place the MSA logo on a publication that is distributed for the event.


\subsection{BPC Funding Outline}
\subsubsection{}
Each semester shall consist of at least two funding cycles.
\subsubsection{}
The Review Board shall recommend allocations to the Assembly,
\subsubsection{}
Any organization may appeal its recommended allocation to the Appeals Board, which shall hear the organization's oral appeal upon request by the organization. 

\subsection{Funding Ineligibility}
\subsubsection{}
BPC shall not fund an organization which is a MSA Committee, Commission, or Select Committee with funds earmarked for BPC.  
\subsubsection{}
An organization may be deemed ineligible for funding by a two-thirds vote of the MSA.

\subsection{Late Applications}
\subsubsection{}
Late applications shall be considered only under extenuating circumstances.
\subsubsection{}
For the BPC Chair to consider a late application, a written statement attached to the funding application must be submitted to the MSA office within three work days of the original application deadline.

\subsection{Violations}
\subsubsection{Student Organization}
\subsubsubsection{}
Any student organization presenting misleading information regarding activities, finances, membership, or any other required information will not have its application considered by BPC and may, upon a majority vote of the MSA, have its student organization status revoked.  

\subsection{Funding Considerations}
\subsubsection{}
Consideration for funding often is based upon the these criteria:
\subsubsubsection{}
Quantity of students affected 
\subsubsubsection{}
The degree of effect on students
\subsubsubsection{}
Effect on the Ann Arbor, University of Michigan, and general Michigan community
\subsubsubsection{}
Effort to receive funding from other sources
\subsubsubsection{}
Completeness of the funding application
\subsubsubsection{}
Unique nature of the event
\subsubsubsection{}
Prior utilization of MSA funding allocations


\section{Community Service Committee (CSC)}

\subsection{}
CSC shall consider funding requests for all student organizations' projects under the following categories, within the guidelines established below:
\subsubsection{}
Direct community service
\subsubsection{}
Indirect community service
\subsubsection{}
Community development
\subsubsection{}
Community organization
\subsubsection{}
Social action
\subsubsection{}
Education

\subsection{Membership Structure}
\subsubsection{}
CSC shall consist of a Chair, Vice-Chair, Reviews, and Appeals Board

\subsubsection{Voting Membership}

\subsubsubsection{}
CSC shall be composed of a Reviews Board and an Appeals Board.
\subsubsubsubsection{}
Each Board must have at least three but no more than nine voting members.
\subsubsubsubsubsection{}
i.	The Chair is non-voting member.
\subsubsubsubsubsection{}
ii.	The Vice-chair is a voting member.
\subsubsubsubsection{}
Each Board must maintain at least one voting MSA Representative.

\subsubsubsection{}
The Reviews Board and Appeals Board shall be recommended by the CSC Chair and shall be confirmed upon a motion, second and a two-thirds majority vote of the Steering Committee (SC) of MSA.

\subsubsubsection{Removal of Voting Members}
\subsubsubsubsection{}
Any voting member from either Board may be removed upon a motion, second, and two-thirds majority vote of the SC.
\subsubsubsubsection{}
The Chair of the SC shall be responsible for contacting the Board member removed with a written explanation of the reason for removal and shall place it in the SC minutes. 

\subsection{Voting Rights}
\subsubsection{}
No voting member from either Board may vote on a request for funds from any student organization that they hold an appointed, compensated, or elected leadership position in.
\subsubsection{}
Violations of paragraph (3.a) shall be grounds for immediate removal from either Board.
\subsubsection{}
Violations by members of the MSA shall constitute malfeasance in office and be grounds for impeachment or removal from all offices and positions held in MSA.
\subsubsection{}
Prior to a vote related to the finances of an organization, members of either Board are required to publicly acknowledge to the respective body any financial interest with that organization. 
\subsubsection{Chair Voting}
\subsubsubsection{}
The Chair shall vote to break a tie. 
\subsubsubsection{}
The Chair may not vote in any other circumstances.

\subsection{CSC Procedure}
\subsubsection{}
CSC shall recommend funding allocations to the MSA on a viewpoint neutral basis. 
\subsubsection{}
CSC may not consider the membership, composition, or political views of any organization when deliberating funding recommendations. 
\subsubsection{}
CSC grant applications shall be made available to student organizations within two weeks of the start of each semester and shall remain available until the application deadline for the final cycle of that semester.
\subsubsection{}
CSC shall consider no more than one application per program per cycle.
\subsubsection{}
Upon the request of an officer of a student organization, the CSC Chair, or designee, shall provide a written justification for that organization's recommended allocation.
\subsubsection{}
Upon the request of any member of MSA, the CSC Chair, or designee, shall provide a written justification for the recommended allocation of any organization. 
\subsubsection{}
Any money allocated to a student organization by the Assembly upon recommendation from CSC which is unspent by the organization shall be considered canceled by the organization and shall revert to MSA.
\subsubsection{}
The CSC Chair, with the assistance of the Administrative Coordinator, will oversee the disbursement and reimbursement process of student organizations from CSC earmarked funds.


\subsection{Student Organization Requirements}

\subsubsection{}
All student groups applying for funding must be registered with MSA and have a valid SOAS account.

\subsubsection{}
Student organizations must present accurate information to CSC through written applications and any oral statements.

\subsubsection{}
CSC conditions

\subsubsubsection{}
CSC may attach any conditions to their allocations regarding the use of funds.

\subsubsubsection{}
Organizations receiving funding must stipulate in a grant agreement that they will adhere to these conditions.

\subsubsubsection{}
Failure to adhere to the conditions attached to the agreement by CSC shall result in a cancellation of the agreement, and all allocated funds shall revert to MSA. 

\subsubsubsection{}
CSC shall not fund, unless deemed necessary by a two-thirds majority vote of the Board reviewing the program:
\subsubsubsubsection{}
Capital goods
\subsubsubsubsection{}
T-shirts
\subsubsubsubsection{}
Travel Costs
\subsubsubsubsection{}
Philanthropic events
\subsubsubsubsection{}
Food not vital to the implementation of the program
\subsubsubsubsection{}
Events that charge admission
\subsubsubsubsection{}
Student salaries for services
\subsubsubsubsection{}
Club sports fees assessed by the Athletic Department
\subsubsubsubsection{}
Projects not open to all UM students

\subsubsubsection{}
Organizations receiving funding from CSC must agree to either include the phrase “Sponsored by the Michigan Student Assembly” or place the MSA logo on a publication that is distributed for the event.

\subsection{CSC Funding Outline}
\subsubsection{}
Each semester shall consist of at least one funding cycle.

\subsubsection{The Review Board}
\subsubsubsection{}
The Review Board shall hold oral interviews for all groups applying for funding before initial allocation decisions are made.
\subsubsubsection{}
At least two Review Board members shall attend each interview and shall take detailed notes of questions, answers, and other pertinent information, to present at the allocation session(s).
\subsubsubsection{}
Following oral interviews, the Review Board shall recommend allocations to the Assembly.

\subsubsection{}
Any organization may appeal its recommended allocation to the Appeals Board, which shall hear the organization's oral appeal upon request by the organization.
 
\subsection{Funding Ineligibility}
\subsubsection{}
CSC shall not fund an organization which is a MSA Committee, Commission, or Select Committee with funds earmarked for CSC.  
\subsubsection{}
An organization may be deemed ineligible for funding by a two-thirds vote of the MSA.

\subsection{Late Applications}
\subsubsection{}
Late applications shall be considered only under extenuating circumstances.
\subsubsection{}
For the CSC Chair to consider a late application, a written statement attached to the funding application must be submitted to the MSA office within three work days of the original application deadline.

\subsection{Violations}
\subsubsection{Student Organization}
\subsubsubsection{}
Any student organization presenting misleading information regarding activities, finances, membership, or any other required information will not have its application considered by CSC and may, upon a majority vote of the MSA, have its student organization status revoked.  

\subsection{Funding Considerations}
\subsubsection{}
Consideration for funding often is based upon the these criteria::
\subsubsubsection{}
Application of areas CSC funds (E.1)
\subsubsubsection{}
Quantity of students affected 
\subsubsubsection{}
The degree of affect on students
\subsubsubsection{}
Affect on the Ann Arbor, University of Michigan, and general Michigan community
\subsubsubsection{}
Effort to receive funding from other sources
\subsubsubsection{}
Completeness of the funding application for funding
\subsubsubsection{}
Unique nature of the event
\subsubsubsection{}
Prior utilization of MSA funding allocations


\section{BPC and CSC Joint Procedures}

\subsection{}
BPC and CSC shall not fund the same expenditures for the same program for the same organization.

\subsection{}
BPC and CSC shall not fund any organizations or programs sponsored or co-sponsored by organizations which have been deemed ineligible for funding by MSA. 


\section{Assembly Funding Recommendations Procedure}

\subsection{}
BPC and CSC funding recommendations shall only be voted on after two reads by the assembly. 

\subsection{}
The assembly may only vote to accept or reject the funding recommendations on a whole.

\subsection{}
A majority vote is necessary to pass BPC and CSC funding recommendations.

\subsection{}
Amendments made to BPC and CSC funding recommendations shall not be in order.


\section{Code of Conduct Advisory Board (C-CAB)}

	
\subsection{Purpose.} The purpose of the Code of Conduct Advisory Board is to review, evaluate and edit the Statement of Students Rights and Responsibilities, the University's codified internal discipline system. 

\subsection{Composition} 

\subsubsection{}
The C-CAB chair will be chosen by a selection committee consisting of the Executive Officers of the Michigan Student Assembly and the Students' Rights Commission Chair(s). The recommendation will be made to the Steering Committee at the meeting preceding summer break for a yearly term beginning in September.

\subsubsection{}
C-CAB will be composed of at least six student members. There is no limit on C-CAB membership, however no member shall vote before she has attended three meetings of the board. Decisions shall be made by a two-thirds vote of the board.

\subsubsection{}
At least three members will represent the Michigan Student Assembly. One of the three representatives must be the Student General Counsel of MSA or her designee. One of the representatives must be the Students' Rights Commission Chair.

\subsubsection{}
Three C-CAB members must be students-at-large. The search and appointment of these members is the duty of the Campus Governance Committee. 

\subsubsection{}
The C-CAB chair must report to the Assembly via written biweekly reports, beginning at the meeting following her appointment. If no report is submitted for two consecutive meeting cycles, the chair will be automatically recalled.

\subsubsection{}
The C-CAB chair shall be confirmed by a majority vote of the Assembly. 

\subsubsection{}
Any changes to the Statement of Students Rights and Responsibilities that will be presented on behalf of the Michigan Student Assembly or C-CAB must be approved by a majority of the Assembly.


\section{MSA Sponsored Activities and Events Guidelines}

\subsection{}
The Michigan Student Assembly shall consider sponsored event requests from all members of the Michigan Student Assembly.
\subsubsection{}
The Michigan Student Assembly shall play an active and significant role in the planning stages, implementation, and operations of all MSA sponsored events.


\subsection{Membership Structure}
\subsubsection{}
All voting members of the Michigan Student Assembly.


\subsection{Voting Rights}
\subsubsection{}
No person voting on funding allocations may be in a compensated position of the organization seeking funding.
\subsubsection{}
Prior to a vote related to the finances of an organization, all voting members are required to declare any financial or personal interest they have with that organization.
\subsubsection{}
Violations of paragraph (5.a) shall result in a hearing by the Central Student Judiciary in a timely fashion.
\subsubsubsection{}
The Student General Counsel shall be responsible for pursuing this matter with the Central Student Judiciary.
\subsubsection{}
Violations by members of MSA shall constitute malfeasance in office and be grounds for impeachment or removal from all offices and positions held in MSA.


\subsection{Procedure for Violation of the Compiled Code or Constitution}
\subsubsection{}
Any voting member of MSA may be removed upon a motion, second, and three-fifths majority vote of the assembly, provided that the reason given violates the aforementioned criteria.
\subsubsection{}
The Counsel of the assembly shall be responsible for contacting the Board member removed with a written explanation of the reason for removal and the Vice President shall place it in the assembly minutes.


\subsection{Sponsored Event Procedure}
\subsubsection{}
The voting members of the Michigan Student Assembly shall make funding allocations on a viewpoint neutral basis. 
\subsubsubsection{}
Funding shall be allocated solely with regard to the extent a particular student group contributes to the educational and social environment of the University and the Mission Statement of the University of Michigan and the Michigan Student Assembly. 
\subsubsubsection{}
This Committee shall not consider political or social ideology or message when allocating funds; however, this shall not preclude the funding of political events.
\subsubsubsection{}
Funds shall not be allocated in a manner that advantages one group over another in terms of its long-term ability to petition student government or influence campus opinion; however, this shall not be construed to limit the committee's ability to fund specific events.
\subsubsubsection{}
Funds shall be used to preserve a marketplace of ideas wherein all students may participate with equal access to resources.
\subsubsubsection{}
Funds shall not be allocated in a manner that considers the membership, composition, or political views of any organization when considering funding requests. 

\subsubsection{}
Funding applications must be introduced through the established guidelines under Meeting Procedures (Article 6, C).

\subsubsection{}
The sponsor and, in the event of an objection, the objector must provide the assembly with one concise sentence, limited to 25 words, as to their reason for their support or objection. 

\subsubsection{}
All funds allotted to an MSA sponsored event shall be withdrawn from the MSA Sponsorship Fund (SF) through a return/receipt reimbursement process.

\subsubsection{}
All funds allocated to an MSA sponsored event which go unspent shall revert back to the MSA Committee Discretionary Account..
\subsubsection{}
The Treasurer, with the assistance of the Administrative Coordinator, will oversee the disbursement and reimbursement process for MSA sponsored events.

\subsection{Application Requirements}
\subsubsection{}
All monies allocated must be utilized through a valid SOAS account including: the MSA Sponsored Activities account or through an MSA Commission or Committee account.

\subsubsubsection{}
Sponsors must present accurate information to the Assembly through written applications and any oral statements.

\subsubsubsection{}
MSA Sponsored Event conditions
\subsubsubsubsection{}
MSA may attach any conditions to their allocations regarding the use of funds.
\subsubsubsubsection{}
Signers on the pertinent SOAS accounts must sign and turn in to the MSA front office staff a grant agreement requiring adherence to MSA conditions prior to the Assembly Decision.
\subsubsubsubsubsection{}
The Assembly will not hear the proposal until the grant agreement has been turned in. 

\subsubsubsubsection{}
Failure to adhere to the conditions attached to the agreement shall result in a cancellation of the agreement, and all allocated funds shall revert to MSA. 

\subsubsubsubsection{}
MSA shall not fund, unless deemed necessary by a two-thirds majority vote:
\subsubsubsubsubsection{}
Capital goods
\subsubsubsubsubsection{}
T-shirts
\subsubsubsubsubsection{}
Newspaper advertisements
\subsubsubsubsubsection{}
Hotel or airfare costs for students traveling from campus
\subsubsubsubsubsection{}
Gas
\subsubsubsubsubsection{}
Food not vital to the implementation of the program
\subsubsubsubsubsection{}
Student salaries for services
\subsubsubsubsubsection{}
Club sports fees assessed by the Athletic Department

\subsection{Appeals Process}
\subsubsection{}
Appeals may be filled only on the grounds of violation of this process or viewpoint neutrality to the Central Student Judiciary. 

\subsection{Violations}
\subsubsection{}
Any sponsor presenting misleading information regarding activities, finances, membership, or any other required information will not have its application considered and may, upon a majority vote of the MSA, have its student organization status revoked.  

\subsection{Funding Considerations}

\subsubsection{}
Consideration for funding will be based upon the these criteria (in no particular order):
\subsubsubsection{}
Quantity of students affected 
\subsubsubsection{}
Degree of effect on students
\subsubsubsection{}
Degree of effect on the Ann Arbor, University of Michigan, and general Michigan community
\subsubsubsection{}
Effort to receive funding from other sources
\subsubsubsection{}
Completeness of the funding application
\subsubsubsection{}
Unique nature of the event
\subsubsubsection{}
Prior utilization of MSA funding allocations, if applicable
\subsubsubsection{}
Degree of MSA involvement in the planning stages, implementation, and operations of the event

\subsection{Procedure to transfer funds from the MSA Sponsored Events Account}
\subsubsection{}
All other motions shall require a 2/3 majority to move funds from the MSA Sponsored Events Account. 
\subsubsection{}
All other motions shall require a 60\% majority to move funds from the MSA Sponsored Events Account.
