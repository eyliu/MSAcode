\article{Student Organizations}

\section{Student Organization Registration.}
A student group seeking registration with MSA must comply with all of the MSA rules and regulations required for student organizations.  Failure to comply with any regulation may result in a termination of the organization's status as a registered student organization.  A student group is registered automatically upon the receipt by the MSA Administrative Coordinator of a qualified application for registration.

\subsection{Requirements for Registration.}
\subsubsection{}
A student organization must have at least five currently-enrolled University of Michigan students as members.
\subsubsection{}
More than half of the total membership of the group must be students currently enrolled at the University of Michigan.
\subsubsection{}
At least two-thirds of the total group membership must be comprised of University of Michigan students, alumni, faculty or staff.
\subsubsection{}
No member of a student organization can receive personal financial benefit from membership in the organization.
\subsubsection{}
No organization can adopt a name which may be construed by the University community as misleading concerning the nature or affiliation of the organization.
\subsubsection{}
``The University of Michigan'' may not be used in the beginning of any student organization name.
\subsubsection{}
A new registration form is required for each school year.
Groups must update MSA with new contacts and authorized signer information as changes occur.
\subsubsection{}
An application for registration must include a written description of the organization.

\subsection{Termination of Registration.}
The registration for all student organizations shall terminate at the end of September of every year.  Registration may also be terminated at any time if the group fails to meet the requirements for registration.  MSA shall have the responsibility of notifying an active student organization of impending termination of its registered status.


\section{Office Space Allocation Committee (OSAC).}
\subsection{Purpose.}
The purpose of the Office Space Allocation Committee is to provide University of Michigan student organizations with criteria and applications for office space and locker usage.  OSAC shall reviews applications for space and allocate office space and lockers on the fourth floor of the Michigan Union.

\subsection{Composition}
\subsubsection{}
OSAC will be composed of 8 student members.  These 8 members constitute the voting members of OSAC.  Quorum shall be a majority of voting committee members.  A simple majority shall be required for all committee decisions.
\subsubsection{}
3 OSAC members will represent the Michigan Union Board of Representatives (MUBR).  One of the three representatives must be the Chairperson of MUBR or her designee.
\subsubsection{}
3 OSAC members will represent the Michigan Student Assembly.  One of the three representatives must be the Vice President of MSA or her designee. 
\subsubsection{}
2 OSAC members will be at-large members.  The selection of these members is the duty of the Campus Governance Committee.
\subsubsection{}
In addition to the 8 voting members, the Administrative Coordinator of MSA and a Michigan Union representative will attend the meetings of OSAC as non-voting members.
\subsubsection{}
The MUBR Chairperson, the MSA Vice President, and the MSA Administrative Coordinator will jointly determine the weekly meeting time and place for OSAC.
\subsubsection{}
If an OSAC member is absent at more than two OSAC meetings, she will be removed from the committee and will automatically be replaced by appointment from the Campus Governance Committee.
\subsubsection{}
Two transition meetings between the old and new OSAC committees will be held.  The first meeting will take place within two weeks of the applications being made available.  The second meeting will occur during the first meeting of the new OSAC in which applications are reviewed.

\subsection{Internal Positions}
\subsubsection{}
The MSA Administrative Coordinator will serve as the chair of OSAC.  During all OSAC meetings, the chair will maintain order within the committee, keep the committee focused, and vote in the event of a tie.
\subsubsection{}
OSAC will appoint an Internal Secretary.  The Internal Secretary will record the minutes from every meeting and keep proper documentation of all activities.  The Administrative Coordinator shall maintain copies of all documentation.
\subsubsection{}
OSAC will also appoint an External Secretary.  The External Secretary will serve as a correspondent to all parties outside the committee.
\subsubsection{}
The Internal and External Secretaries will be elected by the committee through a simple majority of open voting.
\subsubsection{}
All OSAC members must complete a summary of each application they are assigned to review.  These summaries will be maintained by the MSA Administrative Coordinator.

\subsection{Process.}
\subsubsection{}
OSAC application materials shall be made available at the beginning of the winter semester.  
\subsubsection{}
Applications will be due one month after they are made available.
\subsubsection{}
OSAC may contact a student organization for more information or clarification of their application.
\subsubsection{}
No late applications will be accepted.  Student organizations which submit a late application will be notified immediately that their applications were not accepted.

\subsection{Appeals}
\subsubsection{}
Grounds for appeal will be limited to:
\subsubsubsection{}
deviations from the office space allocation procedure as set forth in this article.  
\subsubsubsection{}
penalties applied by MSA, MUBR or the Union Administration regarding office space that are arguably inappropriate for the violation.
\subsubsubsection{}
non-allocation of office space to a student organization who which correctly followed all of the application steps.
\subsubsection{}
The Appeals Board will be composed of 1 MUBR member (not included in the allocation process), 2 MSA members (not included in the allocation process), one Union Administration member (not included in the allocation process), and one student-at-large selected by the Campus Governance Committee.
\subsubsection{}
The composition of the Appeals Board will be determined within the first two weeks that appeals are made available.
\subsubsection{}
An appeal must be submitted in writing, with the president, chairperson, or equivalent's signature, to the MSA office no later than 5 business days after the original penalty was assessed.
\subsubsection{}
The Appeals Board will meet within 2 days of the appeals due date and determine whether the appeal has reason to be heard.
\subsubsection{}
If the Appeals Board finds a reason for appeals to be heard, appeals will take place over the following Saturday and Sunday.  Appeal sign-ups will be posted in MSA.
\subsubsection{}
The organization requesting the appeal can bring no more than 5 members to the appeal.
\subsubsection{}
Only oral presentations with a typed supplement will be considered at the Appeals hearing.
\subsubsection{}
The Appeals Board will decide on the appeal no later than 5 days after the conclusion of the meeting.  The Appeals Board can advise OSAC to reconsider the application, and can ask OSAC to meet with the members of the appealing organization for an information review.
\subsubsection{}
Deviations from the timeline by an appealing student organization will render the appeal null and void.

\section{Ex-Officio Representation}
\subsection{}
A group wishing to attain an ex-officio seat shall submit a list of first name, last name, and email address of at least 400 members, as well as a signed statement acknowledging that they do not belong to a larger organization and are not a college or school student government on the MSA Website's online ex-officio submission tool.
\subsection{}
If a question is raised about the validity of the 400-member roster, the Rules and Elections Committee will conduct an investigation on the number of students in the student organization in question.
\subsection{}
Ex-Officio seats shall expire at the end of every winter semester. Groups wishing to re-apply to retain their seats in the fall shall retain their seat until a determination is made regarding their eligibility for the seat in the fall.
\subsection{}
There shall be no limit to the number of groups allowed to have ex-officio seats. All groups meeting the criteria shall be granted a seat.
\subsection{}
Student organization ex-officio members shall have all the rights of a regular assembly member, except they may not make motions, second a motion, or vote.


\section{Student Organization Funding.}
Student organization funding during the academic year will be determined by the Student Organization Funding Commission (SOFC).  The SOFC shall consider funding requests for all student organizations and their events under the guidelines established below.

\subsection{Leadership.}
The President shall, with the advice and consent of the Assembly, appoint a Chair of the SOFC.  The Chair is a non-voting member.  The SOFC may elect from among their number any other officers they deem expedient.

\subsection{Membership.}
The SOFC must have at least ten (10) but no more than twenty (20) voting members.  At least half of the voting members must be Assembly representatives.  The President shall, with the advice and consent of the Assembly and the SOFC Chair, appoint the members of the SOFC.  The President may remove any member of the SOFC with the written concurrence of three other executives.

\subsection{Schedule.}
Each semester shall consist of at least two funding cycles.  The exact dates of these funding cycles shall be determined by the SOFC Chair.

\subsection{Structure.}
For each funding cycle, the SOFC shall divide its membership into a Reviews Board and an Appeals Board.  The Reviews Board shall recommend allocations to the Assembly.  Any organization may appeal its recommended allocation to the Appeals Board, which shall hear the organization's oral appeal upon request by the organization.  Each Board must have at least five (5) but no more than ten (10) voting members.  At least half of each Board must be voting Assembly representatives.  No voting member of the Reviews Board may serve as a voting member of the Appeals Board within any particular funding cycle.

\subsection{Voting Rights}
\subsubsection{}
No voting member from either Board may vote on a request for funds from any student organization that they hold an appointed, compensated, or elected leadership position in.
\subsubsection{}
Violations of paragraph (3.a) shall be grounds for immediate removal from either Board.
\subsubsection{}
Violations by members of MSA shall constitute malfeasance in office and be grounds for impeachment or removal from all offices and positions held in MSA.
\subsubsection{}
Prior to a vote related to the finances of an organization, members of either Board are required to declare any financial or personal interest they have with that organization.
\subsubsection{Chair Voting}
\subsubsubsection{}
The Chair may vote to break a tie.
\subsubsubsection{}
The Chair may not vote in any other circumstances.

\subsection{Procedure.}
\subsubsection{}
The SOFC shall determine and recommend funding allocations to the MSA on a viewpoint neutral basis.
\subsubsection{}
The SOFC may not consider the membership, composition, or political views of any organization when deliberating funding recommendations.
\subsubsection{}
Funding applications to the SOFC shall be made available to student organizations within two weeks of the start of each semester and shall remain available until the application deadline for the final cycle of that semester.
\subsubsection{}
The SOFC shall consider no more than one application per organization per cycle.
\subsubsection{}
Upon the request of an officer of a student organization, the SOFC Chair, or designee, shall provide a written justification for that organization's recommended allocation.
\subsubsection{}
Upon the request of any member of MSA, the SOFC Chair, or designee, shall provide a written justification for the recommended allocation of any organization.
\subsubsection{}
Any money allocated to a student organization by the Assembly upon recommendation from the SOFC which is unspent by the organization shall be considered canceled by the organization and shall revert to MSA.
\subsubsection{}
The SOFC Chair, with the assistance of the Administrative Coordinator, will oversee the disbursement and reimbursement process of student organizations from SOFC earmarked funds.

\subsection{Student Organization Requirements}
\subsubsection{}
All student groups applying for funding must be registered with MSA and have a valid SOAS account.
\subsubsection{}
Student organizations must present accurate information to the SOFC through written applications and any oral statements.
\subsubsection{Conditions.}
\subsubsubsection{}
The SOFC may attach any conditions to their allocations regarding the use of funds.
\subsubsubsection{}
Organizations receiving funding must stipulate in a grant agreement that they will adhere to these conditions.
\subsubsubsection{}
Failure to adhere to the conditions attached to the agreement by the SOFC shall result in a cancellation of the agreement, and all allocated funds shall revert to MSA.
\subsubsubsection{}
The SOFC shall not fund, unless deemed necessary by a two-thirds majority vote of the committee:
\subsubsubsubsection{}
Capital goods
\subsubsubsubsection{}
T-shirts
\subsubsubsubsection{}
Newspaper advertisements
\subsubsubsubsection{}
Hotel or airfare costs for students traveling from campus
\subsubsubsubsection{}
Gas
\subsubsubsubsection{}
Club sports fees assessed by the Athletic Department
\subsubsubsection{}
Organizations receiving funding the SOFC must agree to either include the phrase ``Sponsored by the Michigan Student Assembly'' or place the MSA logo on a publication that is distributed for the event.
 

\subsection{Funding Ineligibility}
\subsubsection{}
The SOFC shall not fund an organization which is a MSA Committee, Commission, or Select Committee with funds earmarked for SOFC. 
\subsubsection{}
An organization may be deemed ineligible for funding by a two-thirds vote of the MSA.

\subsection{Late Applications}
\subsubsection{}
Late applications shall be considered only under extenuating circumstances.
\subsubsection{}
For the SOFC Chair to consider a late application, a written statement attached to the funding application must be submitted to the MSA office within three work days of the original application deadline.

\subsection{Violations}
\subsubsection{Student Organization}
\subsubsubsection{}
Any student organization presenting misleading information regarding activities, finances, membership, or any other required information will not have its application considered by the SOFC and may, upon a majority vote of the MSA, have its student organization status revoked. 

\subsection{Funding Considerations}
\subsubsection{}
Consideration for funding often is based upon the these criteria:
\subsubsubsection{}
Quantity of students affected
\subsubsubsection{}
The degree of effect on students
\subsubsubsection{}
Effect on the Ann Arbor, University of Michigan, and general Michigan community
\subsubsubsection{}
Effort to receive funding from other sources
\subsubsubsection{}
Completeness of the funding application
\subsubsubsection{}
Unique nature of the event
\subsubsubsection{}
Prior utilization of MSA funding allocations