\article{}

\section{Committee Descriptions.}

\subsection{}
The Budget Priorities Committee (``BPC'') shall review applications for funding from student organizations, and shall submit student organization funding recommendations to the Assembly.

\subsection{}
The Campus Governance Committee (``CGC'') shall
\subsubsection{}
nominate students to University committees;
\subsubsection{}
facilitate communications between MSA and its appointees to University committees;
\subsubsection{}
maintain a list of committees to which MSA makes appointments;
\subsubsection{}
maintain a list of students appointed to University committees;
\subsubsection{}
shall collect mandatory reports from appointees that will be included in the end of semester MSA reports.


\subsection{}
The Rules \& Elections Committee (``R\&E'') shall
\subsubsection{}
monitor and facilitate MSA elections;
\subsubsection{}
record the attendance of MSA members and seek their removal when necessary.
\subsubsection{}
seek appointments to maintain a full Assembly;
\subsubsection{}
make recommendations for changes as needed to the Constitution and Compiled Code.


\subsection{}
The Communications Committee (``Communications'') shall
\subsubsection{}
facilitate all communication between MSA and students;
\subsubsection{}
publicize MSA actions and activities;
\subsubsection{}
advertise MSA services to students;
\subsubsection{}
coordinate MSA press releases;
\subsubsection{}
be responsible for updating and administering the MSA website;
\subsubsection{}
facilitate MSA's presence during New Student Orientation and Welcome Week.
\subsubsection{}
carry out its duties in a fair and equitable manner to all constituents while refraining from promoting the Michigan Student Assembly as a body of bias


\subsection{}
The External Relations Committee (``ERC'') shall
\subsubsection{}
facilitate communication between MSA and individuals and organizations external to the University community;
\subsubsection{}
advocate on behalf of MSA and the student body before organizations external to the University community;
\subsubsection{}
monitor local, state and national government actions concerning MSA and the University;
\subsubsection{}
maintain contact with other college and university student governments and associations of student governments.
\subsubsection{}
Elect a liaison who shall attend Ann Arbor city council meetings and report back to the committee with any information he/she may find pertinent.


\section{Commission Descriptions.}

\subsection{}
The Lesbian, Gay, Bisexual \& Transgender Issues Commission (``LGBT'') shall monitor and work toward the improvement of student life for lesbian, gay, bisexual and transgender students, and shall facilitate communication between relevant student organizations.

\subsection{}
The North Campus Affairs Commission (``NCAC'') shall monitor and work toward the improvement of student life for all students who live and study on North Campus, and shall facilitate communication between relevant student organizations.

\subsection{}
The Minority Affairs Commission (``MAC'') shall monitor and work toward the improvement of student life for all African American, Asian and Pacific American, Latino, Latina, Latin American, Arab American and Native American students, shall strive to educate the university community on minority affairs, and shall facilitate communication between relevant student organizations.

\subsection{}
The Women's Issues Commission (``WIC'') shall monitor and work toward the improvement of student life for all women on campus, shall strive to educate the university community on women's issues and shall facilitate communication between relevant student organizations.

\subsection{}
The Academic Affairs Commission (``AAC'') shall 
\subsubsection{}
monitor and work toward the improvement of the academic experience of all students;
\subsubsection{}
be responsible for the administration of Advice Online.


\subsection{}
The Peace \& Justice Commission (``P\&J'') shall promote a peaceful and equitable University community in a fair and equitable manner to all constituents while refraining from promoting the Michigan Student Assembly as a body of bias.

\subsection{}
The Students Rights Commission (``SRC'') shall
\subsubsection{}
work toward the protection and education of the rights of students;
\subsubsection{}
shall serve on the search committee for and act as a voting member of the Code of Conduct Advisory Board (C-CAB).
\subsubsection{}
assist C-CAB in advocating for the adoption of the recommended changes to the Statement of Students Rights and Responsibilities before the University administration.


\subsection{}
The Environmental Issues Commission (``EIC'') shall
\subsubsection{}
promote a sustainable University community;
\subsubsection{}
promote environmental awareness on campus;
\subsubsection{}
facilitate communication between relevant student organizations.


\subsection{}
The International Student Affairs Commission (``ISAC'') shall monitor and work toward the improvement of student life for all international students on campus, and facilitate communication between relevant student organizations.

\subsection{}
The Health Issues Commission (``HIC'') shall promote the health of all students on campus, educate students on health issues, and facilitate communication between relevant student organizations.

\subsection{}
The Campus Safety Commission (``CSC'') shall promote the safety of all students on campus, educate students on safety issues, and facilitate communication between relevant student organizations.

\subsection{}
The Community Service Commission (``CSC'') shall 
\subsubsection{}
review applications for funding from student organizations engaged in direct community service, community development, and community organizing;
\subsubsection{}
submit community service student organization funding recommendations to the Assembly;
\subsubsection{}
provide financial and organizational training to community service student organizations;
\subsubsection{}
plan, promote, and sponsor events emphasizing the importance of community service to the University community.


\subsection{}
The Voice Your Vote Commission (``VYV'') shall, without engaging in partisan political activity, promote political awareness and participation among the University community.

\subsection{}
The Campus Improvement Commission (``CIC'') shall work to promote and maintain a positive campus atmosphere. They shall also work to bring events that will better the campus as a whole. 

\subsection{}
The Alumni Relations Commission (``ARC'') shall:
\subsubsection{}
build and strengthen the relationship between the Assembly and its 
alumni;
\subsubsection{}
hold a formal event celebrating MSA seniors' graduation each spring
\subsubsection{}
coordinate with the Communications Committee and the Chief of Staff to send a semesterly MSA update / newsletter to the alumni.


\section{Leadership, Composition and Scope of Committees and Commissions.}

\subsection{Leadership.}

\subsubsection{Committee leadership.}
Each committee shall have a single chair and a single vice-chair, both of whom must be voting MSA representatives. 

\subsubsection{Commission leadership.}  Each commission shall have a chair.  Two students may act as co-chairs of a commission.  Commission chairs may, but are not required to, be voting MSA representatives.

\subsection{Composition.}

\subsubsection{}
Unless otherwise specified in the Code, any student, faculty member, or staff member of the University may serve on any MSA committee or commission.
\subsubsection{}
All committees and commissions must re-open applications for membership each semester.

\subsection{Scope.}
No MSA committee or commission shall take or consider action falling outside of its responsibilities as described herein, unless authorized by the Assembly or the Steering Committee.

\section{Select Committees.}

\subsection{Formation.}
The Assembly may, upon two reads and a majority vote, establish one or more select committees.  A motion to form a select committee shall require a second and a written description of the responsibilities of the select committee.

\subsection{Duration.}
Select committees shall expire upon
\subsubsection{}
A date specified in the written description of the committee approved by the Assembly;
\subsubsection{}
The end of the last Steering Committee meeting of a Presidential term after Winter Term elections. 
\subsubsection{}
Two reads and a majority vote by the Assembly.

\subsection{Leadership and Composition.}
The leadership and composition of a select committee shall be analogous to that of an MSA commission.  


\section{Investigative Committees.}

\subsection{Formation.}
The Assembly may, upon two reads and a majority vote, establish one or more investigative committees to investigate the conduct of any MSA member(s), committee(s), or commission(s).  The procedure for the formation of an investigative committee shall be analogous to that of a select committee.

\subsection{Duration.}
The duration of an investigative committee shall be analogous to that of a select committee.

\subsection{Leadership and Composition.}

\subsubsection{}
An investigative committee shall consist of four members, all of whom must be voting representatives on the Assembly.  

\subsubsection{}
The Student General Counsel shall chair the investigative committee.  If the conduct of an Executive Officer is under investigation, the chair of the committee shall be chosen by a majority vote of the Assembly.

\subsubsection{}
The four members of the investigative committee will be chosen by lot from the voting representatives of the Assembly, excluding those whose conduct is under investigation and those who serve as members of any committee(s) and/or commission(s) whose conduct is under investigation.

\subsubsection{}
Any member of any investigative committee who misses two meetings of the committee shall be discharged from the committee, and a replacement shall be selected by the Assembly.


\subsection{Investigative Committee Procedure.}

\subsubsection{}
The quorum of an investigative committee meeting shall be three members.  No meeting of the committee shall be convened without the chair.

\subsubsection{}
The chair of an investigative committee shall vote only to break a tie among the members of the committee.

\subsubsection{}
Investigative committee meetings shall be open to the public.

\subsubsection{}
MSA member(s), committee(s), or commission(s) under investigation shall have the right to attend meetings of the investigative committee, and shall have the right to submit responses to any findings of the committee.

\subsubsection{}
At all stages of an investigation, an investigative committee shall presume that the alleged misconduct did not occur.  Any misconduct must be proven beyond a reasonable doubt.

\subsubsection{}
Upon a finding of misconduct by an investigative committee, the committee may recommend, but its recommendations shall not exceed, the censure or removal of any offending MSA member or officer from his or her chair position, officer position, or MSA membership.

\subsubsection{}
The findings and recommendations of an investigative committee shall be submitted in writing to the Assembly.  The findings of the committee shall be accepted automatically by the Assembly, and may be rejected by the Assembly upon a motion to reject the findings, a second to the motion, and a two-thirds majority vote of the Assembly.

\subsubsection{}
The findings and recommendations of an investigative committee may be appealed to the Central Student Judiciary.


\section{Committee and Commission Chair Elections.}

\subsection{Order of Elections.}
As provided for in the Constitution, the Assembly shall elect the chairs of its committees and commissions each term.  The Assembly shall elect committee chairs first, commission chairs second, and committee vice-chairs third.  The chair elections shall proceed in order of the appearance of the committees and commissions in the Constitution.  The Assembly shall not elect the chair(s) of the International Students Affairs Commission, which shall nominate and elect its chairs internally.  Individuals elected must be approved by the general assembly at large.  The Assembly shall retain the ability to remove the chairs of every committee and commission.


\subsection{Nominations.}

\subsubsection{}
Candidates for chair and vice-chair positions shall be nominated by a motion and a second.

\subsubsection{}
A candidate for a chair or vice-chair position may nominate herself.

\subsubsection{}
Nominations will not be valid unless accepted by the nominated candidate nominated. 

\subsubsection{}
Nominations and acceptances shall be accepted by the chair if made in person.  Nominations and acceptances shall be accepted by the chair via email, telephone, or letter if verified by an executive officer.

\subsubsection{}
The floor shall be closed to nominations when no nominations remain.  The floor may be re-opened to nominations by a motion, second, and majority vote of the Assembly.  


\subsection{Election Procedure.}

\subsubsection{}
Each candidate for a chair or vice-chair position shall have two minutes to address the Assembly.  Candidates shall speak in the order in which they were nominated.

\subsubsection{Questions for candidates.}

\subsubsubsection{}
After all candidates have had the opportunity to address the Assembly, Assembly members shall have an opportunity to ask questions of the candidates.

\subsubsubsection{}
Questions may be addressed to only one candidate, but every candidate shall have the opportunity to answer every question.

\subsubsubsection{}
The Assembly shall ask no more than six questions of every candidate for any one chair or vice-chair position.

\subsubsubsection{}
Candidates shall have thirty seconds each to answer questions from the Assembly.

\subsubsubsection{}
Candidates shall answer questions in the reverse order in which they were nominated.  The order will then iterate respectively.

\subsubsection{Voting by the Assembly.}

\subsubsubsection{}
All elections for chair and vice-chair positions will be by secret ballot.  

\subsubsubsection{}
If only one candidate has been nominated for any position, she shall be automatically elected unless any member of the Assembly objects.  Upon such an objection, the election shall be by secret ballot.

\subsubsubsection{}
A candidate shall be elected if she receives a majority of the votes cast.

\subsubsubsection{}
If no candidate receives a majority of the votes cast, the candidate receiving the lowest number of votes shall be removed from the election and another vote shall be taken.

\subsubsubsection{}
The votes shall be counted by two tellers nominated and approved by a majority of the Assembly.


\section{Committee and Commission Rights and Responsibilities.}

\subsection{Meetings.}
 
\subsubsection{}
Committees and commissions shall meet at least once per week during the fall and winter terms.  This rule shall not apply to the Budget Priorities Committee nor the Community Service Commission.

\subsubsection{}
The chair of each committee and commission shall establish the time and location of each meeting, providing at least one day's notice of the time and location to the Assembly.

\subsubsection{}
Committees and commissions are not required to meet on weeks where the University observes one or more holidays.

\subsubsection{}
Committee and commission chairs or their designees shall be required to attend meetings of the Steering Committee.  

\subsection{Reports.}

\subsubsection{}
Each committee and commission chair shall present a budget to the Treasurer in accordance with IV A 3.

\subsubsection{}
Each committee and commission shall submit a weekly report of its activities to the Assembly. 

\subsubsection{}
Reports may be delivered in person or in writing.
 
\subsubsection{}
A report shall include all information pertinent to the operation of the
committee or commission, and must include an attendance report from the
committee or commission meeting.

\subsection{Attendance.}

\subsubsection{}
The chair of each committee and commission shall be responsible for recording the attendance of meetings of the committee or commission. 

\subsubsection{}
Committee and commission attendance reports shall be submitted to the Vice-Chair of the Rules \& Elections Committee within one day of the committee or commission meeting.


\section{Recall of Committee and Commission Chairs and Vice-Chairs}

\subsection{}
The chair or vice-chair of a committee or commission shall be automatically recalled by the Assembly upon

\subsubsection{}
failure to hold a meeting for two consecutive weeks;
\subsubsection{}
failure to attend two consecutive meetings of the Steering Committee;
\subsubsection{}
failure to submit attendance reports for the committee or commission for two consecutive weeks;
\subsubsection{}
a motion, second, and two-thirds vote of the Assembly.

\subsection{Procedure for the Removal of a Chair or Vice-Chair.}

\subsubsection{}
Any chair or vice-chair who has been recalled by the Assembly shall have the opportunity to address the Assembly for two minutes.

\subsubsection{}
A recalled chair or vice-chair may be reinstated upon a motion, second, and majority vote of the Assembly.

\subsubsection{}
A recalled chair or vice-chair may not be reinstated after a new chair or vice-chair has been elected to fill the recalled position. 

\section{Assembly Member Committee Obligations.}

\subsection{}
Every voting representative on the Assembly must attend at least one MSA committee or commission meeting every week.

\subsection{}
Pursuant to Article II(E)(3)(b) and (c) of the Constitution, the Rules \& Elections Committee shall record one absence for every voting representative on MSA for failure to attend one committee or commission meeting every week.
