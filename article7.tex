\article{}

\section{Summer Assembly.}

\subsection{Description.} During the spring and summer terms the Michigan Student Assembly shall call itself the Summer Student Assembly.  The change in names shall not hinder the work of a member of the Michigan Student Assembly.

\subsection{Composition.} The Summer Student Assembly will be composed of any current voting representatives of the Michigan Student Assembly, and will be chaired by the current executives.

\subsubsection{}
Quorum will be 7 members. These seven members must represent three 
separate colleges. The Vice-President and/or President will be included in this number. 

\subsubsection{}
The provisions of Article VI, Section F of the All-Campus Constitution will apply to members of the Summer Assembly with the exception of recall elections and/or obligations under Article XI.


\subsection{Structure and Guidelines.} 
	
\subsubsection{}
Meetings will be held every two weeks on Wednesdays at 6:00pm beginning with the first Wednesday of the Spring term. Special meetings may be called by the President with 24-hour notice. Attendance will be taken at the beginning and at the end of SSA meetings, however absences will not count against member obligations.

\subsubsection{}
Meetings will be held similar to Michigan Student Assembly meetings. The agenda structure will remain identical. However, the chair must add the following.

\subsubsubsection{Funding Presentations.} Student Groups requesting funding will present their request in a ten minute allotment. This time can be extended with a majority vote.

\subsubsubsection{Closed Session Deliberations.}
After hearing all funding requests for the Summer Student Assembly will convene in a closed session to decide what groups receive funding and the amount disseminated.


\subsubsection{Committees and Commissions.}


\subsection{Jurisdiction.}

\subsubsection{Code and Constitution changes.}
The Constitution cannot be changed during the Summer Student Assembly.  The Summer Student Assembly may modify article VII of the Compiled Code if a compelling interest exists and with a 2/3 majority.  The chair shall e-mail any proposed changes at least forty-eight hours in advance to the Michigan Student Assembly Representatives list.

\subsubsection{Resolutions.}
Resolutions passed by the SSA will become null and void at the first meeting of the Michigan Student Assembly in the Fall term.

\subsubsection{Elections.}
Elections may not be held by the Summer Student Assembly.

\subsubsection{Minutes.}
Minutes will be taken and e-mailed to all members of MSA. A temporary minutes taker will be nominated at the beginning of each meeting.


\section{Budget and Financial Operations.}

\subsection{}
The Summer Assembly only has authority over funds budgeted to it by MSA.
\subsubsection{}
The Summer Assembly will take no action that would affect the budget or
proposed budget of MSA during the fall and winter terms.

\subsection{Student organizations.}
Since BPC does not meeting during spring-summer term,
student organizations will submit funding requests directly to the executive officers for consideration by the Summer Assembly.

\subsubsection{Application Procedures.}
Student organizations may apply to Summer
Assembly for funding of specific events. Summer Assembly funding
applications will be available no later than one week after the start of the
Spring term. Summer Assembly will consider only one application per group.

\subsubsection{Allocation and Disbursement.}
Organizations may not appeal the decision of the Summer Assembly, but must be allowed time to present to the Summer Assembly. The length of the presentation is at the discretion of the Executive Officers, but will be a minimum of five minutes. The officers of the organizations allocated funds will sign a Grant Agreement making them personally liable for repayment of the allocation if stated conditions are not met. Organizations are required to submit accounting to the Administrative Coordinator detailing the use of allocated funds within one month of the conclusion of the event or 14 business days after the first day of classes of the Fall term. The Summer Assembly can attach any conditions to the grant and can specific dates by which the money must be spent. All money not spent by the specified date will revert back to MSA.

\subsection{Reimbursement.}
The amount reimbursed will not exceed the amount allocated by the Summer Assembly and must conform to the categories and any other conditions under which the money was allocated. Money allocated by the Summer Assembly for a project counts as use of MSA resources upon passage by the Summer Assembly.  Student groups who fail to use their money and fail to submit a “Cancellation of Funding” Request by the Second meeting of the Fall term shall be ineligible for funding in following fall term.  The administrative coordinator shall notify each group at least thrice of this requirement.

\subsection{Travel restrictions.}
Money allocated for travel expenses will not cover alcoholic beverages, entertainment, parking tickets, or traffic violations.

\subsection{}
Any disbursement for supplies, capital goods, and salaries deemed necessary for the operation of the office by an executive officer or full-time staff member can be spent.
