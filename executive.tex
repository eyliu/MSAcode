\article{Executive}

\section{The Executive Officers}

\subsection{The President.}
\subsubsection{Executive Power Vested in the President.}
The President shall oversee and coordinate all MSA activities and be the chief spokesperson for MSA unless otherwise specified in the Code or Constitution.
\subsubsection{Appointment powers.}
\subsubsubsection{Executive Committee.}
The President shall have the authority to appoint a Treasurer, Student General Counsel, Chief of Staff, and Chief Programming Officer to the Executive Committee, which shall advise the President on all pertinent matters.  These appointments shall be made with the advice and consent of the Assembly, to be determined by a simple majority vote.  The President may likewise recall these officers with a two-thirds majority vote of the Assembly.  The President may call the Executive Committee into session at any time, and shall serve as its chair in session.
\subsubsubsection{University-wide Committees.}
The President shall also appoint student representatives to university-wide committees.  These appointments shall be made with the advice and consent of the Assembly, to be determined by a simple majority vote.  The President may likewise recall these officers with the written concurrence of three other executives.
	
\subsubsection{Executive Commissions.}
The President may appoint Executive Commissions to study issues on campus, publish reports concerning issues under such purview, and recommend to the Executive Branch such measures as they shall deem appropriate.

\subsubsection{Convening the Legislature.}
The President may call into session the Assembly or the University Council at the President's discretion.

\subsubsection{Non-voting Member of the Assembly.}
The President shall serve as a non-voting ex-officio member of the Assembly.

\subsubsection{Recommend Measures to the Assembly.}
The President and Vice President may, jointly or severally, recommend to the Assembly for its consideration such measures as they shall deem appropriate.

\subsubsection{Reports.}

\subsubsubsection{State of the Students.}
Within the first month of the fall and winter semesters, the President shall submit to the Assembly and to the students at large a report of the state of student government and of the student body.

\subsubsubsection{Transition.}
Before the end of her term, the outgoing President shall prepare a report for her successor to facilitate the transition between administrations.

\subsubsubsection{Regents.}
The President shall make any reports to the University of Michigan Board of Regents available to the Assembly and the students at large before their presentation to the Regents.


\subsection{Vice President.}
\subsubsection{Chairs the University Council.}
The Vice President shall serve as president of the University Council, but shall have no vote, unless the Council shall be equally divided.

\subsubsection{Non-voting Member of the Assembly.}
The Vice President shall serve as a non-voting \textit{ex-officio} member of the Assembly and of any Assembly committee she shall elect.

\subsubsection{Recommend Measures to the Assembly.}
The Vice President may, jointly or severally with the President, recommend to the Assembly for its consideration such measures as they shall deem appropriate.


\subsection{Treasurer.}
The Treasurer shall be the chief financial officer of MSA. The Treasurer and all other officers authorized by the Assembly to disburse funds must be bonded. The Treasurer shall disburse funds appropriated by the Assembly as provided for in this Constitution and in the Compiled Code, and shall create, publish, and maintain a manual to guide student organizations in pursuing budget allocations. The Treasurer shall, at the direction of the President, assist the legislature in drafting a proposed annual budget for the Central Student Government and present it to the Assembly for a vote. The Treasurer may serve as a non-voting \textit{ex-officio} member of any legislative body regarding student finance.

\subsection{Student General Counsel.}
The Student General Counsel shall be the chief representative of the Central Student Government in matters before student judiciaries.  For the purpose of upholding the Constitution and Compiled Code, the Student General Counsel shall have standing for all cases submitted to the Central Student Judiciary.  The Student General Counsel may retain up to three student representatives to serve as assistants in such matters.  The Student General Counsel shall advise the Executive and the Legislature on the interpretation of the Constitution and the Compiled Code, and may serve as a non-voting \textit{ex-officio} member of any legislative body concerning rules and elections of student government.

\subsection{Chief of Staff.}
The Chief of Staff shall oversee attendance and procedural policies at meetings of the Executive Committee and executive commission meetings. The Chief of Staff shall solicit and receive reports of the various organs of government, maintain and publish executive records, and ensure collaboration among the various executive commissions. The Chief of Staff may serve as a non-voting \textit{ex-officio} member of any legislative body concerning rules and elections.

\subsection{Chief Programming Officer.}
The Chief Programming Officer shall serve as principal advisor to the President on matters of student programming, assist executive commissions in the long-range planning and execution of their mandate, and supervise the communications of the Central Student Government. The Chief Programming Officer may serve as a non-voting \textit{ex-officio} member of any legislative body concerning campus communication.


\section{The Executive Committee.}
The Executive Committee shall be comprised of the Executive Officers and the Speaker of the Assembly.  The President may call the Executive Committee into session at any time, and shall serve as its chair in session.


\section{Commissions.}
The following classification scheme for commissions is for organizational purposes only, and is not intended to establish any ranking or hierarchy of commissions or classification of commissions.  All commissions and classifications of commissions are equal under this Code.

\subsection{Logistical Commissions.}

\subsubsection{}
The Student Organization Funding Commission (``SOFC'') shall review applications for funding from student organizations, and shall submit student organization funding recommendations from the Student Organization Funding Account and the Community Service Funding Account to the Assembly.

\subsubsection{}
The Campus Governance Commission (``CGC'') shall
\subsubsubsection{}
assist the President in identifying candidates suitable for nomination to University-wide committees;
\subsubsubsection{}
facilitate communications between MSA and its appointees to University committees; 
\subsubsubsection{}
maintain a list of committees to which MSA makes appointments;
\subsubsubsection{}
maintain a list of students appointed to University committees; and
\subsubsubsection{}
shall collect mandatory reports from appointees that will be included in the end of semester MSA reports.

\subsubsection{}
The External Relations Commission (``ERC'') shall
\subsubsubsection{}
facilitate communication between MSA and individuals and organizations external to the University community;
\subsubsubsection{}
advocate on behalf of MSA and the student body before organizations external to the University community;
\subsubsubsection{}
monitor local, state and national government actions concerning MSA and the University;
\subsubsubsection{}
maintain contact with other college and university student governments and associations of student governments.
\subsubsubsection{}
Elect a liaison who shall attend Ann Arbor city council meetings and report back to the committee with any information he/she may find pertinent.

\subsubsection{}
The Communications Commission (``Communications'') shall
\subsubsubsection{}
facilitate all communication between MSA and students;
\subsubsubsection{}
publicize MSA actions and activities;
\subsubsubsection{}
advertise MSA services to students;
\subsubsubsection{}
coordinate MSA press releases;
\subsubsubsection{}
be responsible for updating and administering the MSA website;
\subsubsubsection{}
facilitate MSA's presence during New Student Orientation and Welcome Week.
\subsubsubsection{}
carry out its duties in a fair and equitable manner to all constituents while refraining from promoting the Michigan Student Assembly as a body of bias.


\subsection{Identity Commissions}

\subsubsection{}
The Diversity Affairs Commission (``DAC1'') shall monitor and work toward the improvement of diversity in student life, and facilitate communication between relevant student organizations.

\subsubsection{}
The Lesbian, Gay, Bisexual \& Transgender Issues Commission (``LGBT'') shall monitor and work toward the improvement of student life for lesbian, gay, bisexual, transgender and ally students, and shall facilitate communication between relevant student organizations.

\subsubsection{}
The Women's Issues Commission (``WIC'') shall monitor and work toward the improvement of student life for all women on campus, shall strive to educate the university community on women's issues and shall facilitate communication between relevant student organizations.

\subsubsection{}
The Minority Affairs Commission (``MAC'') shall strive to educate the university community on issues regarding underrepresented minorities on campus, and shall facilitate communication between relevant student organizations.

\subsubsection{}
The International Student Affairs Commission (``ISAC'') shall monitor and work toward the improvement of student life for all international students on campus, and facilitate communication between relevant student organizations.

\subsubsection{}
The Transfer Student Affairs Commission (``TSAC'') shall monitor and work toward the improvement of student life for all transfer students on campus, and facilitate communication between relevant student organizations.

\subsubsection{}
The Disability Affairs Commission (``DAC2'') shall support and advocate for students with disabilities on campus, and facilitate communication between relevant student organizations.

\subsubsection{}
The North Campus Affairs Commission (``NCAC'') shall monitor and work toward the improvement of student life for all students who live and study on North Campus, and shall facilitate communication between relevant student organizations.

\subsubsection{}
The Greek Relations Committee ("GRC") is responsible for strengthening the relationship between MSA and the Greek community at Michigan. The committee's goal is to ensure that Greek Life is well integrated into the larger University community, to keep campus involved in Greek-sponsored events, and to act as a resource for the Greek community.

\subsection{Issue Commissions}

\subsubsection{}
The Campus Improvement Commission (``CIC'') shall foster communication between MSA and students by promoting opportunities for students to provide ideas that improve life on campus. Also, it shall create positive change on campus by considering and implementing those initiatives that are most important to students.

\subsubsection{}
The Academic Affairs Commission (``AAC'') shall 
\subsubsubsection{}
monitor and work toward the improvement of the academic experience of all students; and
\subsubsubsection{}
communicate with school and college governments, the Senate Advisory Committee on University Affairs (SACUA), and any other University body pertaining to academics.

\subsubsection{}
The Student Safety Commission (``SSC'') shall promote the safety of all students, on or off campus; educate students on safety issues; and facilitate communication between relevant student organizations.

\subsubsection{}
The Environmental Issues Commission (``EIC'') shall
\subsubsubsection{}
promote a sustainable University community;
\subsubsubsection{}
promote environmental awareness on campus;
\subsubsubsection{}
facilitate communication between relevant student organizations.

\subsubsection{}
The Health Issues Commission (``HIC'') shall promote the health of all students on campus, educate students on health issues, and facilitate communication between relevant student organizations.

\subsubsection{}
The Peace \& Justice Commission (``P\&J'') shall promote a peaceful and equitable University community in a fair and equitable manner to all constituents while refraining from promoting the Michigan Student Assembly as a body of bias.

\subsubsection{}
The Students Rights Commission (``SRC'') shall 
\subsubsubsection{}
work toward the protection and education of the rights of students;
\subsubsubsection{}
shall serve on the search committee for and act as a voting member of the Code of Conduct Advisory Board (C-CAB).
\subsubsubsection{}
assist C-CAB in advocating for the adoption of the recommended changes to the Statement of Students Rights and Responsibilities before the University administration.

\subsubsection{}
The Voice Your Vote Commission (``VYV'') shall, without engaging in partisan political activity, promote political awareness and participation among the University community.


\section{Commission Composition and Leadership.}

\subsection{Composition.}
Unless otherwise specified in the Code, any student, faculty member, or staff member of the University may serve on any MSA commission.

\subsection{Commission Chairs.}
The President shall, with the consent of a simple majority of the Assembly, appoint Commission chairs, who shall not be considered officers of the Central Student Government.  The President may remove a
Commission chair with the written concurrence of three other executives.

\subsection{Other Commission Officers.}
Members of the Commission may elect from among their number any other officers they deem expedient.


\section{The Cabinet.}

The Cabinet shall be comprised of the Executive Committee and the Commission Chairs.  The President may call the Cabinet into session at any time, and shall serve as its chair in session.