\article{General Provisions}

\section{Title.}
This Compiled Code of the Michigan Student Assembly, enacted pursuant to Article II, Section 2, of the Constitution of the Student Body of the Ann Arbor Campus of the University of Michigan, contains all regulations, excluding provisions of the Operating Procedures, currently and permanently affecting student government or the student body.


\section{Definitions.}
As referenced in this Compiled Code, the following terms shall have meaning as defined in this section.

\subsection{}
"Constitution" shall mean the Constitution of the Student Body of the Ann Arbor Campus of the University of Michigan.

\subsection{}
"Michigan Student Assembly" or "MSA" shall mean the central student government of the University of Michigan established by the Constitution.

\subsection{}
"University" shall mean the University of Michigan.

\subsection{}
"Campus" shall mean the Ann Arbor campus of the University.

\subsection{}
"Student organization" shall mean a student organization explicitly recognized by MSA.

\subsection{}
"Assembly" shall mean the Student Assembly defined in Article II, Section 2 of the Constitution.

\subsection{}
"Central Student Judiciary" or "CSJ" shall mean the student judicial body of the University established by the Constitution.

\subsection{}
"Compiled Code" or "Code" shall mean this document.

\subsection{}
"Student" shall mean a person enrolled at the University, or a person enrolled in the University during the previous full term who is eligible to be enrolled in the subsequent full term.


\section{Amendments to the Compiled Code.}

\subsection{}
Any amendment to the Code must be read twice by the Assembly. The amendment must have been discussed or originated in the Rules and Elections Committee prior to its introduction, or it shall be immediately committed to the Rules and Elections Committee with instructions to return within three weeks time with their recommendations for it. Amendments to the Code or ballot questions to amend the Constitution may not be considered by the Assembly until at least one week has elapsed since the time of introduction.

\subsection{}
An amendment to the Code or ballot question to amend the Constitution shall not be considered by the Assembly unless both the pre-amendment language and the proposed amended language have been made available to the Assembly in the same document.

\subsection{}
An amendment to the Code or ballot question to amend the Constitution shall not be considered by the Assembly if the amendment creates a conflict within the Code or a conflict between the Code and the Constitution.

\subsection{}
Amendments to the Code shall have immediate effect, but shall be ineffective if not recorded in the MSA minutes of the meeting at which they were enacted by the Assembly. Amendments to the Constitution, if adopted by the student body, shall be effective as provided for in the Constitution.

\subsection{}
Amendments to the Code shall pass by the Assembly with a simple majority vote.


\section{Conflict of Law.}

\subsection{}
Any ambiguity between the Code and the Constitution shall be resolved to give full effect to the Constitution.  

\subsection{}
Any ambiguity between current or pending MSA legislation and the Code shall be resolved to give full effect to the Code.

 

