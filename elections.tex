\article{Elections}

\section{Definitions.}

\subsection{}
``Election Code'' shall mean Article VI of the Compiled Code.

\subsection{}
``Candidate'' shall mean a person seeking office in an election, and a President -- Vice President pair seeking those offices in an election.

\subsection{}
``Candidate-Elect'' shall mean any eligible student selected to run in an Election on the Election ballot.

\subsection{}
``Campaign'' shall mean supporting, endorsing, advertising, or aiding the election of any candidate.

\subsection{}
``Candidate's Meeting'' shall mean a preliminary meeting that all interested election candidates must attend before the campaign period starts. The meeting is mandatory for all those who apply for candidacy and wish to appear on the election ballot. Failure to attend the meeting may result in an automatic assessment of 1 demerit for the candidate, to be given by the University Elections Commission.

\subsection{}
``Demerit'' shall mean a mark awarded against a candidate and/or party for fault or offence \todo{\textit{sic}} that is in violation of the Election Code. Any candidate who accrues 5 or more demerits will be removed from the election, and any party that receives 10 or more demerits will be automatically removed from the election. Demerits will be assessed by the University Elections Commission.

\subsection{}
``Referendum'' shall mean any referendum, initiative, recall, or constitutional amendment to be voted upon by students in an election.

\subsection{}
``Party'' shall mean a group of candidates for President, Vice President, or representative identified by a common party name on the election ballot.

\subsection{}
``Days before the start of the election'' shall mean the number of days before the first day on which voting is scheduled to occur.

\subsection{}
``Complaint'' shall mean any document delivered to the Election Director alleging a violation of any rule in the Election Code.

\subsection{}
``Student-at-large'' shall mean any student not currently a representative, commission chair, an executive, a member of the University Elections Commission, the election director or select committee chair on MSA, nor a candidate seeking office in an election.

\subsection{}
``Email'' shall mean any piece of digital communication sent by a candidate, candidate's agent, a party, or a party's agent and received by another individual. Emails sent to groups, aliases, or listservs will be counted once per recipient.


\section{Election Schedule.}  

\subsection{Election Dates.}
The Assembly shall schedule two annual elections, one in the fall semester and one in the winter semester.  The Rules \& Elections Chair shall recommend to the Assembly the dates on which to schedule the elections.  Each election must be held for two consecutive weekdays occurring no earlier than five weeks before the last day of classes for each semester.  
\subsection{Election Deadlines.}
\subsubsection{}
No later than 42 days before the start of the election, the Student General Counsel will submit her nominations for Election Director and University Elections Commission to the University Council.
\subsubsection{}
No later than 30 days before the start of the election, the Election Director shall make candidacy applications available in the MSA office and shall begin advertising the MSA election.
\subsubsection{}
No later than 31 days before the start of the election, the Assembly may approve any amendments to the Election Code.
\subsubsection{}
No later than 5:00 pm 16 days before the start of the election, candidates-elect must file their candidacy applications with the Election Director, Administrative Coordinator, MSA Rules and Elections Committee Chairs, or full/part time staff employed by the University for MSA purposes. The Election Director may set the filing date prior to 5:00 PM 16 days before the start of the election.
\subsubsection{}
No later than 16 days before the start of the election, the Election Director shall hold a required meeting of all candidates and the campaign period shall commence at the close of the meeting.
\subsubsection{}
No later than 12 days before the start of the election, an official sample ballot will be posted on the voting website and in the MSA office.
\subsubsection{}
No later than 12 hours after the end of the election, the Election Director shall deliver unofficial results to all candidates, current MSA Representatives and Executives, CSJ Justices, and the Michigan Daily via electronic mail.
\subsubsection{}
No later than 18 hours after the Election Director delivers unofficial results for the election, any election grievances must be delivered to the Election Director.
\subsubsection{}
No later than 24 hours after any decision of the University Elections Commission, any appeal of that University Elections Commission decision must be delivered to CSJ.
\subsubsection{}
At the first Steering Committee meeting after the end of the election, the Election Director shall announce official election results.
\subsubsection{}
At the first Assembly meeting following the Steering Committee meeting at which official election results are announced, the term of incumbent representatives shall expire and the term of newly-elected representatives shall commence. 


\section{Election Staff.}

\subsection{Election Director.}

\subsubsection{Eligibility.}
The Election Director must be a currently-enrolled University student and not a member of CSJ, nor a representative, executive officer, commission chair, or select committee chair on MSA, nor a candidate in any election during which she will also serve as Election Director.
\subsubsection{Appointment.}
The Student General Counsel shall appoint an Election Director with the advice and consent of the University Council.  A majority vote shall be required to confirm the nomination.  If the nomination is rejected by the University Council, the appointment process shall recommence.  
\subsubsection{Removal.}
Any member of the Assembly or of the University Council may seek the removal of the Election Director, who shall be removed by a two-thirds majority vote of the University Council.  If the Election Director is removed by the University Council, the appointment process shall recommence.
\subsubsection{Duties.}
\subsubsubsection{}
The Election Director shall make weekly reports to the Assembly beginning the week following her confirmation and ending the week after the election ends.
\subsubsubsection{}
The Election Director shall consult the Office of the Registrar to verify the enrollment status of all candidates and ensure that all candidates fulfill the requirements of the Constitution and of the Election Code.
\subsubsubsection{}
The Election Director shall advertise the MSA election in coordination with the Communications Committee, Voice Your Vote Commission, the Rules and Elections Committee, the University Elections Commission, the Assembly, and the University Council.
\subsubsubsection{}
The Election Director shall prepare and make available in the MSA office candidacy applications.
\subsubsubsection{}
Candidates shall be informed of any Election Code changes made by the Assembly after candidacy applications are available.
\subsubsubsection{}
The Election Director shall schedule, preside at, and announce at least 48 hours prior to its commencement, a meeting of all candidates. 
\subsubsubsection{}
The Election Director shall be responsible for ensuring the correct operation of the voting website and the candidate information website.
\subsubsubsection{}
The Election Director shall be available in person, by phone, or by some means of electronic communication during the election period, and shall promptly respond to any questions received from candidates.
\subsubsubsection{}
The Election Director shall preside over meetings of the University Election Commission as well as the the University Election Judiciary.

\subsection{University Elections Commission.}

\subsubsection{Composition.}
The University Elections Commission shall be composed of at least five enrolled students, including at least one member from the Assembly and at least one member from the University Council.  No candidate may serve on the University Elections Commission. 

\subsubsection{Appointment.}
The Student General Counsel shall submit nominations for membership on the University Elections Commission to the Steering Committee, which shall submit the nominations to the Assembly for confirmation.  The Assembly may approve all, none, or any of the nominations, and may amend the composition of the University Elections Commission.  Confirmation of the University Elections Commission shall be upon a motion, second, and majority vote of the Assembly.

\subsubsection{Removal.}
The University Council may, by a two-thirds vote, remove any member of the University Elections Commission.  If a removal from the University Elections Commission results in an University Elections Commission membership that does not meet the requirements of the Election Code, the appointment process shall recommence.

\subsubsection{Meetings.}
The University Elections Commission shall meet as necessary.  Meetings shall be scheduled with at least 24 hours advance notice by the Election Director.

\subsubsection{Duties.}
\subsubsubsection{}
The University Elections Commission shall assist the Election Director in fulfilling her obligations and executing the election.
\subsubsubsection{}
The membership of the University Elections Commission will also comprise the membership of the University Elections Judiciary, which will be the body which hears and decides upon all election complaints.  The University Elections Judiciary may be convened by the Election Director with less than 24 hours advance notice.

\subsection{Backup Election Director.}

\subsubsection{Eligibility.}
The University Elections Commission shall elect a Backup Election Director from among its own membership.
\subsubsection{Duties.}
\subsubsubsection{}
The Backup Election Director will serve as a voting member of the University Elections Commission and shall serve as the Secretary of the Board.
\subsubsubsection{}
The Backup Election Director will serve temporarily as the Election Director in such instances where asked to do so by the Election Director or when the Election Director is unable to fulfill her duties.
\subsubsection{Removal.}
Any member of the Assembly or of the University Council may seek the removal of the Backup Election Director, who shall be removed by a two-thirds majority vote of the University Council.  If the Election Director is removed by the University Council, the University Elections Commission shall elect a new Backup Election Director.


\section{Election Publicity.}
\subsection{}
All elections conducted by MSA must be advertised to students.
\subsection{}
The Election Director, with the assistance of the University Elections Commission, must send at least one email to all enrolled students advertising, at minimum, the election dates, voting website address, and hours of operation of the voting website.


\section{Candidacy Applications, and Candidate and Party Names.}
\subsection{Candidacy Applications.}
\subsubsection{Contents.}
\subsubsubsection{Personal Application.}
The candidacy application shall contain a personal application that shall require every candidate to provide her name as it is to appear on the ballot, her current local address, her current local telephone number, her email address, her UM ID number, her school(s) of enrollment, and her school of candidacy. 
\subsubsubsection{Receipt.}
The candidacy application shall contain a receipt, which shall be signed by the Election Director, Rules and Elections Chair, Rules and Elections Vice Chair, or Administrative Coordinator upon receipt of the candidacy application and returned to the candidate for verification.
\subsubsubsection{Party Application.}
The candidacy application shall contain a party application which shall require candidates who wish to run in a party to set forth the name of the party, and the name and dated signature of every candidate wishing to run in that party.  A party need only submit a single party application. 
\subsubsubsection{Signatures.}
Every application submitted to the Election Director or Administrative Coordinator must bear the signatures and dates of signatures of every candidate named in the application.
\subsubsubsection{Candidate Oath.}
Every candidate-elect will sign a statement attesting to the fact that all information provided by the candidate-elect is truthful to the best of her knowledge and that she was an enrolled student at the University of Michigan's Ann Arbor campus by the end of the third week of the semester containing the election in question.
\subsubsubsection{Informative Material.}
The candidacy application shall contain informative material which may be retained by the candidate.  At a minimum, this material must include: an election calendar with appropriate deadlines clearly marked; a complete list of positions to be elected; a Housing Department application for door-to-door solicitation in residence halls; a copy of the rules regarding elections and canvassing in Residence Halls; information on how to access the ITS acceptable use policies; a copy of the Election Code; and information regarding the registration of candidates with the online voting system.
\subsubsection{}
The Candidates packet will be jointly prepared by the Election Director, the Chairman of the Rules and Elections Committee, and the Student General Counsel.

\subsection{Candidate and Party Names.}
\subsubsection{}
Candidates who choose to run in a party will be identified on the ballot by their common party name.
\subsubsection{}
Candidate and party names must be fully written out, with the exception of common abbreviations, with the first letter of each word capitalized, with the exception of articles, connectors, and prepositions, and the remainder of every word in the party name in lower case.
\subsubsection{}
A party name may be in all upper-case letters if it is an acronym.
\subsubsection{}
No party name may be longer than 100 characters, including spaces and punctuation.
\subsubsection{}
No party name may consist solely of or begin with the word ``independent''.
\subsubsection{}
No candidate may use a name on the ballot that is not her own.  A candidate who wishes her nickname to appear on the ballot may spell her nickname in between her real first and last names.
\subsubsection{Previously Used Party Names.}
No party may choose the name of another party that was properly filed in any election within four years prior to the current election without the written authorization of a majority not greater than five of the candidates who ran with that previous party.
\subsubsection{Deceptive Party Names.}
No party may use a deceptive party name.
\subsubsubsection{}
Party names shall be posted by the Election Director immediately after the deadline for filing candidacy applications.  
\subsubsubsection{}
Challenges to party names must be submitted to the Election Director within 24 hours of the posting of the registered party names.
\subsubsubsection{}
The University Elections Commission shall decide whether a party name is deceptive, and if it so finds shall allow the party 24 hours in which to submit an alternate party name.
\subsubsubsection{}
Replacement party names may also be challenged.
\subsubsection{Size Limitation.}
No party may run more candidates for any school or college than there are seats available to be elected from that school or college.
\subsubsection{Conflicting Applications.}
No candidate shall run with more than one party.  Any candidate who signs more than one party application shall not be placed on the ballot as a candidate. No candidate shall run simultaneously as an independent and with a party.
\subsubsection{}
In the process of randomizing party and candidate names on the online voting ballot, independents shall be grouped together and randomized as if they were another party.

\subsection{Withdrawal of Candidacy.}
Any candidate may withdraw from the election by submitting a written request to the Election Director no later than 8 days prior to the election.  A candidate who withdraws from the election but is nonetheless elected shall have the status of a resigned member of the Assembly.
\subsection{Simultaneous Candidacies.}
Candidates may not run for more than one electable MSA position simultaneously.

\section{Demerit System}
\subsection{}
All Campaigns to serve on the Michigan Student Assembly shall be subject to the rules and regulations found in this article of the Compiled Code.
\subsection{}
At any point after the official start of the campaign period, the election director may with the approval of the University Elections Commission assess demerits to individual candidates and / or parties for the violations listed within this article of the Compiled Code.
\subsection{}
Any candidate who accrues 5 demerits in a specific election will be automatically removed from the election.
\subsection{}
Both candidates and parties may appeal the assessment of demerits to CSJ.
\subsection{}
The University Elections Judiciary may assess demerits outside of the guidelines specified in this article if it finds sufficient cause to do so given by mitigating factors, extreme circumstances, or a lack of intent on the part of the accused.


\section{Campaign Rules.}

\subsection{Campaign Period.}
The campaign period should commence immediately following the Candidates' Meeting with the Election Director, no later than 16 days before the start of the election. Campaign rules shall apply from the start of the official campaign period until the newly elected representatives are seated.  The existence of the official campaign period shall not prohibit candidates from campaigning before the campaign period.

\subsection{University Policies.}
The Election Director shall encourage all candidates to read and become familiar with all relevant university and residence hall policies that may be affected by their campaigns.  The University Elections Commission may only hold candidates responsible for adhering to the Election Code, and may not hold candidates responsible for violations or alleged violations of any university policy not listed in the Election Code.

\subsection{MSA Endorsements Prohibited.}
Neither the Assembly nor any of its committees, commissions, select committees, University Elections Commission, University Elections Judiciary, nor Election Director shall endorse any candidate in any election.  As individuals, members of MSA may endorse the candidacy of any candidate in any election.  Members of the University Elections Commission may not endorse the candidacy of any candidate or party.

\subsection{Campaign Rules.}

\subsubsection{Minor Infractions.}
Any violation will result in the assessment of one to two demerits.
\subsubsubsection{Identification.}
All printed campaign material must be identified, at minimum, by a statement in the form: ``Paid for by <address>'', where <address> is a valid email address of the candidate or party.  Buttons and clothing are exempt from this rule.  A violation shall be considered for every 50 pieces of campaign material per day.  No more than four violations may be assessed within 24 hours of notifying the candidate.
\subsubsubsection{Prohibited Posting Areas.}
No campaign materials may be affixed on or in any University building.  Residence halls and designated posting areas in University Buildings are excepted from this rule. No more than one violation may be assessed per day.
\subsubsubsection{Destruction of Campaign Material Prohibited.}
No candidate may move or obscure the campaign material of another candidate or party.  A student removing campaign material from her private property is not in violation of this rule.
\subsubsubsection{Implying Elected Incumbency.}
No printed campaign material for any candidate may imply incumbency if the candidate is not a current representative on MSA.  Candidates appointed to MSA may use the word ``retain'' on their printed campaign material but may not use the word ``re-elect''.  No more than one violation may be assessed per day.
\subsubsubsection{Not Attending a Mandatory Candidates' Meeting.}  Candiates \todo{\textit{sic}} wishing to be placed on the ballot and having submitted a complete candidacy application on time that fail to attend a mandatory candidates' meeting shall be in violation of this rule. Candidates may not be found to be in violation of this rule more than once per election cycle.

\subsubsection{Major Infractions.}
Any violation will result in the assessment of two to four demerits.
\subsubsubsection{Unauthorized Endorsement.}
Any campaign material claiming endorsement from any person or group of people that is not authorized by that person or group of people must include a disclaimer in the form: ``Not authorized by <name>'', where name is the name of the person or group of people from whom endorsement is claimed.  Candidates and parties may imply endorsement by securing and retaining written permission from the person or group of people from whom endorsement is claimed. No more than one violation may be assessed per day.
\subsubsubsection{Destruction of Campaign Material Prohibited.}
No candidate may destroy, deface, remove, or alter the campaign material of another candidate or party. A student removing campaign material from her private property is not in violation of this rule.
Influencing a Student While Voting Prohibited.  No candidate may influence any student while the student is voting. The mere presence of a candidate in the vicinity of a voter while voting shall not constitute a violation of this rule. 
\subsubsubsection{Inappropriate and irresponsible use of email privileges prohibited.}
No party or candidate may knowingly send an unsolicited electionic \todo{\textit{sic}} communication or email to members of the University Community. The following actions will also be prohibited under this rule: harvesting addresses from the University of Michigan online directory, running mass-mail programs, sending campaign email to individuals that are not students, and sending campaign email to groups or email lists that the sender does not own.

\subsubsection{Egregious Infractions.}
Any violation will result in the assessment of at least 4 demerits.
\subsubsubsection{Defacement Prohibited.}
No campaign material may be affixed to any surface that would be permanently and seriously damaged by the campaign material or the material used to affix or attach the campaign material. No campaign material may be affixed to paint or glass in any University building.
\subsubsubsection{Preventing Voting Prohibited.}
No candidate may prevent any student from lawfully voting.
\subsubsubsection{Bribery Prohibited.}
No candidate may promise or offer compensation, monetary or otherwise, in exchange for vote(s).  Campaign pledges shall not constitute violations of this rule. The distribution of campaign material to voters shall not constitute a violation of this rule.
\subsubsubsection{Fraudulent Voting Prohibited.}
No candidate may cast any ballot on behalf of another student. No candidate may log into the voting website using any uniqname that is not her own.


\section{Penalties for the Violation of Campaign Rules.}
\subsection{Jurisdiction.}
The University Elections Judiciary shall hear cases involving the alleged violation of any campaign rule, and shall meet to determine whether demerits should be assessed against any candidate(s) or party(ies).
\subsection{Exclusivity of Campaign Rules.}
No single piece of campaign material may violate more than one campaign rule.  All campaign rules shall be mutually exclusive.  No candidate or party may be in violation of more than one campaign rule for a single act or campaign material.
\subsection{Assessment of Demerits.}
\subsubsection{}
Demerits will be assessed based on their classification as described in Section G above.
\subsubsection{}
The University Elections Judiciary may assess demerits outside of the guidelines specified in this article if it finds sufficient cause to do so given by mitigating factors, extreme circumstances, or a lack of intent on the part of the accused.
\subsubsection{Violations by a Candidate.}
If the University Elections Judiciary determines that a candidate has violated a campaign rule and decides to assess demerits against that candidate, the University Elections Judiciary may only assess demerits against that specific candidate.
\subsubsection{Violations by a non-Candidate.}
\subsubsubsection{}
If the University Elections Judiciary determines that a campaign rule has been violated by someone other than a candidate and decides to assess demerits for the violation of the rule, the University Elections Judiciary must first determine whether or not the rule was violated by a person working in coordination with a candidate, more than one candidate, or a party.
\subsubsubsection{}
If the University Elections Judiciary determines that the campaign rule was violated by a person working in coordination with only one candidate, the University Elections Judiciary may assess demerits only against that specific candidate.
\subsubsubsection{}
If the University Elections Judiciary determines that the campaign rule was violated by a person working in coordination with more than one candidate, the University Elections Judiciary must assess the demerits at full value against all offending candidates.
\subsubsubsection{}
If the University Elections Judiciary determines that the campaign rule was violated by a person working in coordination with a party, the University Elections Judiciary must assess the demerits at full value against all candidates of the party.\todo{Note that this is very different than the \textit{status quo}, and might be reverted to the \textit{status quo}}

\subsection{Election Complaint Procedures.}
\subsubsection{Receipt and Disbursement.}
\subsubsubsection{}
Any student may file a complaint with the Election Director alleging a violation of the campaign rules.  Upon receipt of the complaint, the Election Director shall immediately deliver copies of the complaint to all of the named respondents, to the members of the University Elections Judiciary, to the Chair of the Rules \& Elections Committee, to the Student General Counsel, and to the Chief Justice of the Central Student Judiciary.
\subsubsubsection{}
Neither the Election Director nor any member of the University Elections Commission may file a complaint with the Election Director.
\subsubsubsection{}
Complaints must set forth the names of the respondent(s), the salient facts upon which the complaint is based, and clearly identify the campaign rule that has been allegedly violated.
\subsubsection{Withdrawal.}
At any time during the complaint process, the petitioner of the complaint may withdraw the complaint.  Upon withdrawal, the complaint is canceled and may not be heard by the University Elections Judiciary.  A complaint that has been withdrawn may not be reinstated.
\subsubsection{Submission of Respondents Brief.}
A respondent need not submit a written brief, but may file such a written brief within 24 hours of her receipt of the complaint.  Failure to respond in writing shall not waive the respondent's right to defend herself against the allegation.

\subsubsection{Preliminary Hearing.}
\subsubsubsection{}
Within 24 hours of receipt of the respondent's brief, or the expiration of respondent's 24-hour deadline, the Election Director shall hold a preliminary hearing.  The petitioner and respondent shall both be notified of the date, time, and location of the preliminary hearing, which shall be open to the public.  The preliminary hearing may not commence without the attendance of a quorum of the University Elections Judiciary.
\subsubsubsection{}
At the preliminary hearing, the petitioner shall have ten minutes to present an oral argument in support of the complaint, after which the University Elections Judiciary may ask questions of the petitioner and, if present, the respondent.
\subsubsubsection{}
Prior to the conclusion of the preliminary hearing, the University Elections Judiciary may order an investigation into the allegations raised in the grievance. This investigation may be performed by members of the University Elections Commission or designated members of the University community with specific areas of expertise relevant to the investigation, as seem fit by the University Elections Judiciary. Results of any Judiciary-ordered investigation will be made known to all parties and shall be concluded prior to a full hearing by the Board.
\subsubsubsection{}
After the preliminary hearing, the University Elections Judiciary shall retire to a meeting, which shall be open to the public, at which the University Elections Commission shall decide whether the complaint is (a) likely to be true, and (b) if true, would result in the assessment of any demerits.  The complaint process shall not proceed unless the University Elections Judiciary finds both elements to exist.
\subsubsubsection{}
After the University Elections Judiciary meeting, the Election Director shall notify the petitioner and respondent in writing of the University Elections Judiciary decision, and shall, if necessary, schedule a hearing to take place within 24 hours of the preliminary hearing.

\subsubsection{Burden of Persuasion.}
At all stages of the complaint process, the University Elections Judiciary and CSJ shall assume that the allegations set forth in the complaint are not true.  At all stages beyond the preliminary hearing , the petitioner shall have the burden of proof of showing that the allegations set forth in the complaint are true beyond a reasonable doubt.

\subsubsection{Hearing.}
\subsubsubsection{}
The hearing shall not commence without the attendance of a quorum of the University Elections Judiciary.
\subsubsubsection{}
The petitioner will be given five minutes to make an opening statement in support of the complaint, after which the respondent will be given five minutes to make an opening statement against the complaint.
\subsubsubsection{}
The petitioner shall present her case first, and shall have thirty minutes to make a case in support of the complaint.  The respondent shall then present her case, and shall have thirty minutes to make a case against the complaint.
\subsubsubsection{}
The petitioner shall be given ten minutes to make a closing argument in support of the complaint, after which the respondent shall be given ten minutes to make a closing argument against the complaint.
\subsubsubsection{}
After the hearing, the University Elections Judiciary shall retire to a meeting.  The decision of the University Elections Judiciary must be written, and must be delivered to the petitioner and the respondent within 36 hours of the hearing.
\subsubsubsection{}
Failure of the University Elections Judiciary to reach a decision in the matter shall result in a cancellation of the complaint, which shall not be further pursued by the University Elections Judiciary.  Failure of the University Elections Judiciary to deliver a written opinion to the petitioner and respondent within 36 hours of the hearing shall result in a cancellation of the complaint, which shall not be further pursued by the University Elections Judiciary.

\subsubsection{Removal.}
Any candidate against whom five or more demerits have been assessed shall be removed from the election.  

\subsubsection{Warning.}
The University Elections Judiciary may find a candidate or party in violation of the campaign rules but nonetheless assess no demerits against the candidate or party.

\subsubsection{Appeals.}
The respondent and/or petitioner may appeal any decision of the University Elections Judiciary to CSJ.  


\section{Post-Election Procedure.}
\subsection{Eliminating Derogatory Write-In Votes.}
Immediately following the completion of the election, the University Elections Commission shall review the election results and eliminate any write-in responses they deem to be inappropriate and/or offensive.
\subsection{Release of Results.}
Unofficial results, with derogatory write-in votes deleted but noting the number of derogatory write-in votes that were removed, are to be released to candidates and parties no later than 24-hours after the completion of the election. Official results, noting the number of write-in votes deemed derogatory and removed, shall be posted on the MSA website immediately after being approved by MSA Steering.
\subsection{Seating of New Members.}
Newly elected members and officers of MSA will begin their term of office at the regular Assembly meeting to occur at 7:30 P.M. on the first Tuesday following the University Council meeting at which the official election results are announced.  MSA officers and members will remain in office until the seating of their successors (unless removed from office by methods specified in the All-Campus Constitution).  The President will, before beginning his/her term in office, swear to affirm the following oath: ``I promise to faithfully execute the office of Michigan Student Assembly President.''  This oath will be administered by the Chief Justice of the Central Student Judiciary.  The Executive Vice President will, before beginning his/her term of office, swear to affirm the following oath: ``I promise to faithfully execute the office of Michigan Student Assembly Vice President.''  This oath will be administered by the Associate Chief Justice of the Central Student Judiciary.
\subsection{Appeals in Progress.}
While appeals to CSJ are being pursued, the decision of the University Elections Judiciary and/or Election Director is in force unless CSJ stays their decision.
\subsection{Debriefing the Assembly.}
The Election Director shall debrief the Assembly of the election no later than two weeks following the completion of the election. If there is an appeal in progress, the debrief shall occur at the next MSA General Assembly meeting once the appeal has been settled.

\section{Petitions and Ballot Questions.}
This section applies to all questions placed on the ballot in an MSA election.  All restrictions applying to candidates also apply to anyone campaigning for a ballot question.  However, in cases of conflict, this section supersedes the Election Code.

\subsection{Amending this Section.}
Amendments to this section must be approved by MSA at a regular MSA meeting occurring at least seven days after the regular MSA meeting at which the amendment was first introduced.  No amendment approved less than 30 days before an election may apply to that election.

\subsection{}
A ballot question is any referendum, initiative, referral or recall question or constitutional amendment question (regardless of method of initiation) to be voted upon in an election.

\subsection{}
Any ballot question to be placed on the ballot must be submitted to the Election Director at least 25 days before the election.  The Election Director will notify CSJ of any ballot questions submitted by MSA or by petition.  

\subsubsection{}
In the case of a petition, two copies of the petition, including the original document, shall be submitted to the Election Director, for distribution to the MSA Program Manager and CSJ.

\subsection{}
CSJ will examine each ballot question at a hearing no later than 16 days before the election to verify that the ballot question complies with the provisions of the Compiled Code and the MSA Constitution, is worded in a manner that is accurate, fair, concise, and reflective of the content of the amendment or legislation (or meets the requirements for a recall question), and (in the case of petitions) is in the proper form. CSJ can only bar a referendum question which fails to meet these requirements; it cannot bar a question from the ballot because it dislikes the legislative goals. Any appeal of the CSJ decision must be filed within 24 hours of the decision, and CSJ will resolve the appeal no later than 14 days before the election.

\subsection{Form of petitions.}
All petitions for ballot questions will be in the form outline below.  A petition sponsor should consult with R\&E or the SGC if he/she has any questions concerning the proper form of a petition.  Responsibility for complying with the provisions of this Code falls upon the sponsor, and ignorance, error, misinterpretation or mistake of law is not an excuse for failure to comply.
\subsubsection{Title.}
The title of the petition will be stated entirely in uppercase letters at the top of each page of the petition.
\subsubsection{Text.}
Following the title, the petition will contain the full and exact text of the question. The question must be worded in a manner that is accurate, fair, concise, and reflective of the content of the amendment or legislation (or meet the requirements for a recall question).
\subsubsection{Signatures.}
Below the full text on each page of the petition will appear the words, ``We, the undersigned currently enrolled students, petition for a campuswide vote on the proposal above.''  Each petition will have a column for the signature of the student, his/her printed name, his/her student identification number, and his/her uniqname.
\subsubsection{Circulator's statement.}
At the bottom of each page of a petition there will be the following statements: ``I have circulated this petition and believe all of the signers to be currently-enrolled students.''  The petition will be signed by the circulator with his/her printed name, uniqname, and date upon which the petition was circulated.  The petition will also state the names of official sponsors of the petition.
\subsubsection{Distribution of signatures.}
Any question to be placed on the ballot by petition must obtain the support of 1000 currently-enrolled students at the University of Michigan, Ann Arbor.  No more than 80\% of the signatures can be from one school or college. 
\subsubsection{Certification of petitions.}
CSJ, with the election staff, will examine each petition for a ballot question, verifying whether the petition has met the requirements stipulated above.  Student status must be verified by checking no less than 100 of the uniqnames online or with the Registrar's Office.
\subsubsection{Validity of a petition.}
Parties to any action challenging the validity of a petition will be allowed to inspect the petition document.


\section{Seat Apportionment.}
Describes the method and manner in which seats will be apportioned among schools and divided between terms.
\subsection{Unit Apportionment.}
\subsubsection{}
Only ``constituent degree-granting units'' (i.e. any school, college, or academic division located at the Ann Arbor campus of the University of Michigan that is also authorized to recommend to the board of Regents the granting of degrees as specified in Chapter IX and Chapter XI of the Bylaws of the Board of Regents) will receive seats on the Assembly.  Students in non-granting units will be represented by the school which authorizes their degree (their constituent degree-granting unit).
\subsubsection{}
Each degree-granting unit will receive one representative for each 800 students or major fraction thereof enrolled in the unit.  Each degree-granting unit will receive at least one representative on the Assembly.
\subsubsection{}
The most currently available fall and winter term enrollment data will be averaged for determining enrollment.  The data comes from the Office of the Registrar's Term Enrollment and Credit Hour Reports; specifically, the ``102-Enrollment by Unit, Gender, Class Level'' report. 
\subsubsection{}
The apportionment process will take place during the winter semester prior to the commencement of elections.
\subsubsection{}
Seats shall be apportioned according to the total number of students listed in the report for each unit excluding graduate students that receive their degrees from Rackham.  Rackham seats shall be apportioned according to the total number of graduate students that receive their degrees from that school.

\subsection{Academic Term Apportionment}
\subsubsection{}
All full-term seats will be apportioned to the March election.
\subsubsection{}
Any seats that are vacant or held by appointment will be up for election as half-term seats in the November election.


\section{Department of Public Safety Oversight Committee Elections.}
\subsection{}
All-campus elections for the two representative seats on the Department of Public Safety Oversight Committee shall be held in concurrence with the November and March elections.
\subsection{}
That \todo{\textit{sic}} each election shall seat a student on the DPS Oversight Committee for a period of 1 year, with the runner-up acting as the backup Representative should the elected Representative resign.
\subsection{}
The rules and procedures for this election shall follow the same rules and procedures outlined for Michigan Student Assembly elections.

